\documentstyle{article}
\def\fat{\framebox[1mm]{\rule{0mm}{2mm}}}
\def\CSP{{\sc csp}}
\def\RPC{{\sc rpc}}
\def\SR{{\sc sr}}
\def\co{{\bf co}}
\def\oc{{\bf oc}}
\def\pa{{$\parallel$}}
\def\lb{\langle}
\def\rb{\rangle}
\def\ra{$\rightarrow$}
\def\await{{\bf await}}
\def\zimplies{\Rightarrow}
%
\topmargin 0pt
\oddsidemargin 17pt \evensidemargin 17pt
\textheight 8.3in %
\textwidth 6.0in

\parindent=0pt
\pagestyle{empty}
\begin{document}


{\bf CS 
\large \bf 730 Distributed Computing Principles\\[5pt]
\large Exam 2\\[10pt]
Nov 26, 1996 \quad 99 points max \quad 75 minutes\\
}
\bigskip
{Mateti,  Fall Quarter 1996, Wright State U}\\[-5pt]
\hrule

\begin{enumerate}

\item (8*5 points)
Explain/Discuss/Dispute/Suggest/Answer, in a {\em few} lines, the following.

\begin{enumerate}
\item Dispute:
A {\em critical assertion} is one without which a program will abort.


\item Explain the four bold-faced terms in the following
attempt at defining what a ``safety'' property is.  Let {\sc bad} be a
{\bf predicate} characterizing a  ``bad'' {\bf state} of program code
segment S.  Assume that {\bf \{P\} S \{Q\}} holds.  Assume that I is a
{\bf global invariant}.  If I $\zimplies$ $\neg$ {\sc bad}, we say
that $\neg$ {\sc bad} is a safety property for S.

\item
``Some day, if not at the end of {\sc cs} 730, you will appreciate the
book {\em Concurrent Programming} by Andrews.''  Explain if this is a
safety, or liveness property, or perhaps neither.

\item
Discuss:
\RPC\ cannot be used for peer-to-peer distributed computing.

\item
Dispute: The \RPC\ runtime locates a server for each invocation
of a server operation.

\item
Suggest a mechanism similar to RPC but in the context of Linda.


\item
Discuss:
A typical machine (node) in a distributed system can run several
concurrent processes.  Each of these processes must maintain its own
logical clock in order to correctly deal with the {\em happened
before} relation -- even if all the processes are running
on no more than two nodes.

\item
In the Andrews' distributed semaphore implementation of our textbook
that we discussed, there are numerous {\sc ack} messages being sent.
Discuss if these are unnecessary.
\end{enumerate}


\item (19 points)
Describe a distributed termination algorithm, with or without tokens.
The process graph is a binary tree.

\item (20*2 points)
In both items below $a_i$ is the same as {\tt a[i]}.\\

(a) Assume that the Linda tuple space is loaded with tuples $\lb
``nt'', n_t \rb$, $\lb ``nd'', n_d \rb$, $\lb ``d'', d \rb$, $\lb
``p'', p \rb$ so that $0 \le n_d \le n_t$, and $\lb ``s'', i, a_i
\rb$, for $1 \le i \le n_t$, so that the $a_i$ sequence is sorted in
the non-decreasing order.  In this sequence there are {\em at least}
$n_d$ occurrences of $d$ which need to be removed.  Develop a C-Linda
solution for this removal and adjustment of indices $i$ for numbers
beyond the deleted $d$'s.  At the end, the tuple space contains the
$a_i$ after the removal of $d$'s.  Use $p$ processes.  Assume that $p
<< n_t$.

(b) Now in \CSP.  The values $a_i$, for $1 \le i \le n_t$, are now
held by processes $P_i$, one per process, as in the Small Set of
Integers.  Initially, the $a_i$ sequence is sorted in the
non-decreasing order.  $P_0$ sends $d$ to $P_1$, which sends it to
$P_2$, etc.  Write a \CSP algorithm so that all $d$ are ``removed''.
Note that no $n_d$ is given.  At the end,
the $a_i$ sequence should be sorted in the
non-decreasing order.
\end{enumerate}
\end{document}
