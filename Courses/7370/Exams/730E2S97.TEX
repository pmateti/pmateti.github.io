\documentclass[12pt]{article}
\def\fat{\framebox[1mm]{\rule{0mm}{2mm}}}
\def\pr{$\parallel$}
\def\rar{$\rightarrow$}
\def\CSP{{\sc csp}}
\def\RPC{{\sc rpc}}
\def\SR{{\sc sr}}

\parindent=0pt
\begin{document}


{\bf CEG 
{\large \bf 730 Distributed Computing Principles}\\[5pt]
Final Exam\quad June 12, 1997\quad 100 points max \quad 120 minutes}\\
{Mateti,  Spring Quarter 1997, Wright State U}\\[-5pt]
\hrule

\begin{enumerate}

\item (10*5 points)
Explain, in a few lines, the truth or falsity of the following
statements.

\begin{enumerate}

\item
Any solution to the distributed mutual exclusion problem can be
thought of as a solution to the problem of implementing semaphores in
a distributed system, and vice versa.

\item
A {\em critical assertion} is one without which a program will abort.

\item
A ``safety'' property is defined thus: Let {\sc bad} be a {predicate}
characterizing a ``bad'' {state} of program code segment S.
Assume that {\{P\} S \{Q\}} holds.  Assume that I is a {global
invariant}.  If I $\Rightarrow$ $\neg$ {\sc bad}, we say that $\neg$
{\sc bad} is a safety property for S.

\item
``Some day, if not at the end of {\sc ceg} 730, you will appreciate
and marvel at the book {\em Concurrent Programming} by Andrews.''
Explain if this is a safety, or liveness property, or perhaps neither.



\item
Asynchronous message passing cannot be implemented using synchronous
message passing primitives.

\item
The \CSP\ code {\tt [X!2 \pr Y!3]} is equivalent to \\
{\tt
[  true \rar\ X!2; Y!3 \fat\ true \rar\ Y!3; X!2]}

\item
The ``agenda'' paradigm of distributed computing is also known as
``probe/echo''.

\item
In \SR, when invoking an operation implemented by a {\tt proc} through
a {\tt send} the effect is that of a rendezvous.

\item
In an \SR system running on Unix, each creation of a {\tt vm()}
results in an execution of {\tt rsh} and a new Unix process.

\item
Java RMI is RPC applied in the context of the Java language.

\end{enumerate}

\item (25 points)
Given: A bag $B$ of integers, and the size $b$ of it.  It is known
that each element in the bag appears exactly $d$ times.  We wish to
delete the duplicates resulting in a set of only $b/d$ distinct
elements that occurred in the bag.  It is guaranteed that $d > 0, b >
0$.  Unfortunately, the value $d$ is unavailable and needs to be
computed by examining the bag $B$.

Write a solution in C-Linda.  Fully explain it.  Assume that the tuple
space already contains {\tt <"B", i>} for all {\tt i} in $B$, and the
size of the bag {\tt <"nB", b>}.  You lose 2 points for {\sl each} use
of {\tt inp} or {\tt rdp}.  As usual, we wish to maximize concurrency,
we prefer a symmetric solution, and correctness rates much much higher
than efficiency.

%\newpage
\item (15 points)
The following is claimed to be an implementation of {\bf region $v$
when $B$ do $S$ end}.  Either give a convincing argument that the
above algorithm is correct, or give a scenario demonstrating a
situation that violates a requirement of the problem.

{\bf
\begin{tabbing}
00\=00\=00\=\kill
semaphore $e := 1$, $g := 0$, $d := 0$ initially;\\
integer $nd := 0$, $ng := 0$ initially;\\
\\
P$(e)$;\\
while not $B$ do\+\\
  $nd := nd + 1$;\\
  if $ng > 0$ then V$(g)$ else V$(e)$ fi;\\
  P$(d)$;\\
  P$(g)$;\\
  $ng := ng - 1$\-\\
od;\\
$S$;\\
while $nd > 0$ do\+\\
  V$(d)$;\\
  $nd := nd - 1$;\\
  $ng := ng + 1$\-\\
od;\\
if $ng > 0$ then V$(g)$ else V$(e)$ fi
\end{tabbing}
}

\item (10 points)
Udding's solution to the starvation-free mutual exclusion problem is
displayed on the board.  Suppose none of the semaphores used obey the
so-called ``weak semaphore assumption.''  Would this algorithm remain
a valid solution?  If so, give a convincing argument.  If not, exhibit
a scenario demonstrating a situation that violates a requirement of
the problem.
\end{enumerate}
\end{document}
