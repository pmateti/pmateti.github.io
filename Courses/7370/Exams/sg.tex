\documentstyle[12pt]{article}
\def\fat{\framebox[1mm]{\rule{0mm}{2mm}}}
\def\pr{$\parallel$}
\def\rar{$\rightarrow$}
\def\CSP{{\sc csp}}
\def\RPC{{\sc rpc}}
\def\SR{{\sc sr}}

\parindent=0pt
\begin{document}


{\bf CEG 
\large \bf 730 Distributed Computing Principles\\[5pt]
\large Final Exam\\[10pt]
\quad 100 points max \quad 120 minutes\\
}
\bigskip
{Mateti,  Fall Quarter 1992, Wright State U}\\[-5pt]
\hrule

\begin{enumerate}

\item (5*5 points)
Explain, in a few lines, the truth or falsity of the following
statements.

\begin{enumerate}
\item 
The ``standard'' solution \quad {\tt P(m); CS; V(m);} \noindent is not
starvation-free only in the distributed systems context, but is
ok when the system is a single-cpu system.

\item
Any solution to the distributed mutual exclusion problem can be
thought of as a solution to the problem of implementing semaphores in
a distributed system, and vice versa.

\item
In \SR, when invoking an operation implemented by a {\tt proc} through
a {\tt send} the effect is that of a rendezvous.

\item
Asynchronous message passing cannot be implemented using synchronous
message passing primitives.

\item
The \CSP\ code {\tt [X!2 \pr Y!3]} is equivalent to \\
{\tt
[  true \rar\ X!2; Y!3\\
\fat\ true \rar\ Y!3; X!2\\
]}

\end{enumerate}

\item (30 points)
Write an algorithm in C-Linda for the following problem.  Explain your
solution fully.  Given, in the tuple space {\sc ts}, a bag $B$ of
integers {\tt <"B", i>} for all {\tt i} in $B$, and the size of the
bag {\tt <"nB", nb>}, compute the second largest number {\tt <"mx2nd",
x>} in the bag.

\item (10 points)  Rewrite the above algorithm in \CSP.

\item (20 points)
Give a simulation of the Dining Philosophers, with or without a
servant, in \SR.


\newpage
\item (15 points)
The following is claimed to be an implementation of {\bf region $v$
when $B$ do $S$ end}.

{\bf
\begin{tabbing}
00\=00\=00\=\kill
semaphore $e := 1$, $g := 0$, $d := 0$ initially;\\
integer $nd := 0$, $ng := 0$ initially;\\
\\
P$(e)$;\\
while not $B$ do\+\\
  $nd := nd + 1$;\\
  if $ng > 0$ then V$(g)$ else V$(e)$ fi;\\
  P$(d)$;\\
  P$(g)$;\\
  $ng := ng - 1$\-\\
od;\\
$S$;\\
while $nd > 0$ do\+\\
  V$(d)$;\\
  $nd := nd - 1$;\\
  $ng := ng + 1$\-\\
od;\\
if $ng > 0$ then V$(g)$ else V$(e)$ fi\\
\end{tabbing}
}

\noindent
Either give a proof that the above algorithm is correct, or give a
scenario demonstrating a situation that violates a requirement of
the problem.

\end{enumerate}
\end{document}
