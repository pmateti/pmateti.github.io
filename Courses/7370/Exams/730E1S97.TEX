\documentclass[12pt]{article}
\def\fat{\framebox[1mm]{\rule{0mm}{2mm}}}
\def\CSP{{\sc csp}}
\def\RPC{{\sc rpc}}
\def\co{{\bf co}}
\def\oc{{\bf oc}}
\def\pa{{$\parallel$}}
\def\lb{$\langle$}
\def\rb{$\rangle$}
\def\ra{$\rightarrow$}
\def\await{{\bf await}}
\def\zand{\wedge}\def\zor{\vee}	\def\znot{\neg}

\parindent=0pt
\begin{document}


{\bf CEG 
\large \bf 730 Distributed Computing Principles\\[5pt]
\large Mid Term Exam\\[10pt]
May 6, 1997 \quad 100 points max \quad 75 minutes\\
}
\bigskip
{Mateti,  Spring Quarter 1997, Wright State U}\\[-5pt]
\hrule

\begin{enumerate}

\item (7*5 points)

Explain/Discuss/Dispute/Answer, in a {\em few} lines, the following.

\begin{enumerate}

\item
Discuss the semantics of the fat-bar operator as in {\tt S1 \fat\ S2 }.

\item 
The ``standard'' solution \quad {\tt P(m); CS; V(m);} \noindent is not
starvation-free {\em only} in the distributed systems context, but is OK
when the system is a single-CPU system.


\item
The little-endian-ness of a machine has no relevance to \RPC.

\item
What does \verb|svc_run()| do?  Why does it not ``return''?

\item
Explain what {\em dispatching} is, in the context of \RPC.

\item
A procedure designed to be called remotely {\em must not} have
(the equivalent of Pascal's) var parameters.

\item
``Every philosopher will eventually get hungry.''  Explain
if this is a safety, or liveness property, or perhaps neither.
\end{enumerate}


\item (5+5+10+10 points)
\begin{enumerate}
\item
Explain the meaning of the notation: $ \{ P \} \quad S \quad \{ Q \} $

\item
Explain the meaning of the notation:
\[\frac{H_1, H_2, \ldots , H_n}{C}\]


\item
Explain the {\bf Concurrency Rule}, page 67, inference rule 2.11, of
Andrew's book reproduced below.\\

\begin{centering}
\{$P_i\} S_i \{Q_i\}$ are interference-free theorems, $1 \leq i \leq
n$\\[-8pt]
~.\hrulefill\\[-4pt]
$\{P_1 \zand \ldots \zand P_n\}$
\quad\co\quad $S_1 \parallel \ldots \parallel S_n$ \quad\oc\quad
$\{Q_1 \zand \ldots \zand Q_n\}$\par
\end{centering}

\item
Consider the following program segment:

\begin{tabbing}
\co\= \lb \await0\= $x \geq 3$ \=\ra0\= $x := x - 3$ \rb\kill
\co\> \lb \await \> $x \geq 3$ \>\ra \>$x := x - 3$ \rb\\
\pa\> \lb \await \> $x \geq 2$ \>\ra \>$x := x - 2$ \rb\\
\pa\> \lb \await \> $x = 1$ \>\ra \>$x := x + 5$ \rb\\
\oc
\end{tabbing}

Let $P$ be a predicate that characterizes the weakest deadloc-free
precondition for the program, i.e., the largest set of states such
that, if the program is begun in a state satisfying $P$, then it will
terminate if scheduling is weakly-fair.  Determine $P$.  Explain your
answer.
\end{enumerate}


\item (5*7 points)
Udding's solution to the starvation-free mutual exclusion problem is
reproduced below.
\begin{enumerate}
\item State the weak semaphore assumption.  
\item Must each of the three semaphores satisfy this assumption?  
\item Explain what would happen if we exchange that {\tt V(eu)} of line 5
with {\tt V(qu)} of line 8.  \item Explain what would happen if we
exchange that {\tt V(eu)} of line 12 with {\tt V(qu)} of line 8.
\item Explain what would happen if we delete {\tt P(qu)} from line 2 and
{\tt V(qu)} from line 8.
\end{enumerate}

\begin{verbatim}
 0 init ne := nm := 0; eu := 1; mu := 0; qu := 1;
 1 P(eu); ne := ne+1; V(eu);
 2 P(qu); P(eu);
 3    nm := nm+1;
 4    ne := ne-1;
 5    if ne > 0 --> V(eu)
 6    [] ne = 0 --> V(mu)
 7    fi;
 8 V(qu);
 9 P(mu); nm := nm-1;
10    <cs>
11    if nm > 0 -->  V(mu)
12    [] nm = 0 -->  V(eu)
13    fi
\end{verbatim}

\end{enumerate}
\end{document}
