\documentstyle{article}
\def\fat{\framebox[1mm]{\rule{0mm}{2mm}}}
\def\CSP{{\sc csp}}
\def\RPC{{\sc rpc}}
\def\SR{{\sc sr}}
\def\co{{\bf co}}
\def\oc{{\bf oc}}
\def\pa{{$\parallel$}}
\def\lb{\langle}
\def\rb{\rangle}
\def\ra{$\rightarrow$}
\def\await{{\bf await}}
%\textheight=8.3in
\parindent=0pt
\pagestyle{empty}
\begin{document}


{\bf CEG 
\large \bf 730 Distributed Computing Principles\\[5pt]
\large Final Exam\\[10pt]
June 8, 1994 \quad 100 points max \quad 120 minutes\\
}
\bigskip
{Mateti,  Spring Quarter 1994, Wright State U}\\[-5pt]
\hrule

\begin{enumerate}

\item (7*5 points)
Explain, in a few lines, the (degree of) truth or falsity of the
following statements.

\begin{enumerate}
\item Give a rigorous definition of interference-free.

\item Describe the weakly fair scheduling policy.

\item
\RPC\ cannot be used for peer-to-peer distributed computing.

\item
The \RPC\ runtime locates a server for each invocation
of a server operation.

\item
The \CSP\ code {\tt [X!2 \pa\ Y!3]} is equivalent to \\
{\tt
[  true \ra\ X!2; Y!3\\
\fat\ true \ra\ Y!3; X!2\\
]}

\item
Why is it that invoking an \SR\ operation via ({\tt send} and {\tt
call}) produces one of four different interprocess communication
effects can not be justified rationally.

\item
There is no way to implement a ``strong'' semaphore in Linda.

\end{enumerate}

\item (3*10 points)
There are two kinds of processes, the $X$'s and the $Y$'s, that enter
a ``room''.  An $X$ process cannot leave until it meets at least two
$Y$ processes, and a $Y$ process cannot leave until it meets at least
one $X$ process.  Once a process has met the required number of other
kind of processes, that process is free to leave the room.  Develop
and explain three solutions: (i) using conditional critical regions
only, (ii) using semaphores only, and (iii) in CSP.

\item (15 points)
Describe a ``sensible'' scheme, involving new stub generators as well
as required or recommended styles of coding, for \RPC\ based
client-servers to have a few (say under 10) global integer variables.

\item (20 points)
Assume that the Linda tuple space is loaded with tuples $\lb ``nt", n_t
\rb, \lb ``nd", n_d \rb, \lb ``d", d \rb, \lb ``p", p \rb$ so that $0 \le
n_d \le n_t$, and $\lb ``s", i, a_i \rb$, for $1 \le i \le n_t$, so
that the $a_i$ sequence is sorted in the non-decreasing order.  In
this sequence there are at least $nd$ occurrences of $d$ which need to
be removed.  Develop a C-Linda solution for this removal and
adjustment of indices $i$ for numbers beyond the deleted $d$'s.  Use
$p$ processes.

\end{enumerate}
\end{document}
