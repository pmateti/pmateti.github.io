\documentstyle[12pt]{article}
\def\CSP{{\sc csp}}
\def\SR{{\sc sr}}
\def\RPC{{\sc rpc}}
\def\fat{\framebox[1mm]{\rule{0mm}{2mm}}}
\def\co{{\bf co}}
\def\oc{{\bf oc}}
\def\pa{{$\parallel$}}
\def\lb{$\langle$}
\def\rb{$\rangle$}
\def\ra{$\rightarrow$}
\def\await{{\bf await}}

\parskip=0pt
\parindent=0pt
\begin{document}


{\bf CEG 
\large \bf 730 Distributed Computing Principles\\[5pt]
\large Final Exam\\[10pt]
Nov 29, 1993 \quad 100 points max \quad 120 minutes\\
}
\bigskip
{Mateti,  Fall Quarter 1993, Wright State U}\\[-5pt]
\hrule

\begin{enumerate}

\item (6*5 points)
Explain/Discuss/Answer, in a few lines, the following.

\begin{enumerate}

\item
Discuss the required invariant for the Readers and Writers (with or
without priority for writers, your choice) problem.


\item Give a rigorous definition of interference-free.

\item Describe the weakly fair scheduling policy.

\item
Given asynchronous message passing primitives, called {\tt asend()}
and {\tt arecv()}, build the synchronous message passing primitives
called {\tt ssend()} and {\tt srecv()}.

\item
Describe the effect of {\tt send()}-ing to an operation
implemented as a {\tt proc} in \SR.

\item
Implement a binary semaphore named $s$, initially set to 1, in Linda.

\end{enumerate}


\item (6*5 points)
The following items deal with the  client/server paradigm.  


\begin{enumerate}

\item
In the context of \RPC, 
describe the tasks performed by the stub
procedures in the client and in the server.

\item
In the context of \RPC, 
decsribe a scheme for locating the servers.

\item
In the context of \RPC, 
why is it the case that a remotely callable procedure must not
have a global variable?

\item
Explain the terms: big endian, exactly once, at most once, at least
once, and idempotent.

\item
Sketch how client/server paradigm can be implemented in Linda.

\item
Why is the client/server paradigm so popular in our times?


\end{enumerate}


\item (20 points)
Write an algorithm in Linda for the following problem.  Given, in the
tuple space {\sc ts}, a bag $B$ of integers {\tt <"B", i>} for all
{\tt i} in $B$, and the size of the bag {\tt <"nB", nb>}, compute the
second largest number {\tt <"secondmaxB", ? mx2>} in the bag.
The number of workers to use is given as {\tt <"nprocesses", p>}. 
Assume that there are at least two distinct numbers in $B$.
Maximize concurrency.
Explain your algorithm fully.

\item (5+5+10 points)
Study the following code as a `solution' for the star\-vation-free
mutual exclusion problem.  Make the same assumption regarding weak
semaphores that [Udding 86] does.  Obviously, {\tt s[0]} and {\tt
s[1]} are semaphores.  The integer {\tt z} is a variable shared by all
processes, whereas integers {\tt x} and {\tt y} are local to each.
Note that the critical section appears twice.

\begin{minipage}{2in}
\begin{verbatim}
global var z : int := ...;
var x : int := ...;
    y : int := ...;

x := z;
P(s[x]);
  if z = x then
    y := 1 - z;
    P(s[y]);
      z := y;
      repeat
        b := true;
        V(s[x]);
        P(s[x]);
      until b;
      <critical section>
    V(s[y])
  else
    b := false;
    <critical section>
  fi;
V(s[x])
\end{verbatim}
\end{minipage}
\begin{minipage}{3in}
\begin{enumerate}
\item
List the initial values that make sense for all the semaphores, and
integers.

\item Are the semaphores binary, or general?
Give a rigorous argument supporting your answer.

\item
Either give a convincing argument that the above algorithm is correct,
or give a scenario demonstrating a situation that violates the
requirements of the problem.

\end{enumerate}
\end{minipage}
\end{enumerate}
\end{document}
