\documentstyle[12pt]{article}
\def\fat{\framebox[1mm]{\rule{0mm}{2mm}}}
\def\pr{$\parallel$}
\def\CSP{{\sc csp}}
\def\RPC{{\sc rpc}}
\parindent=0pt
\begin{document}


{\bf CEG 
\large \bf 730 Distributed Systems I\\[5pt]
\large Mid Term Exam\\[10pt]
Oct 24, 1991 \quad 100 points max \quad 75 minutes\\
}
\bigskip
{Mateti,  Fall Quarter 1991, Wright State U}\\[-5pt]
\hrule

\begin{enumerate}

\item (6*5 points)
Explain, in a few lines, the truth or falsity of the following
statements.

\begin{enumerate}

\item One difference between a weak semaphore {\tt w} and a strong
semaphore {\tt s} is that {\tt V(s); P(s)} is equivalent to a no-op
whereas {\tt V(w); P(w)} is not.

\item
The little-endian-ness of a machine has no relevance to \RPC.

\item
A procedure designed to be called remotely {\em must not} have
(the equivalent of Pascal's) var parameters.

\item
Any safety property can be expressed as a liveness property.

\item It is possible for a small box, say like that of a PC, to
contain a distributed system.

\item
The \CSP\ code {\tt [X!2 \pr Y!3]} is equivalent to \\
{\tt
[  true --> X!2; Y!3\\
\fat\ true --> Y!3; X!2\\
]
}

\end{enumerate}

\item (25 points) We are faced with the problem of controlling the
two-way traffic on an old bridge.  The bridge is so old and weak that
it can only hold at most 10 cars at a time.  Trucks and other heavy
vehicles are forbidden.  Traffic lights, and sensors that detect the
passing of cars are installed on both ends of the bridge.  Present all
the processes in \CSP.  State clearly any other assumptions that you
found necessary for your solution.

\item (10+10 points)
(a) Explain the client/server paradigm.  (b) In the context of \RPC,
describe the tasks performed by the stub procedures in the client and
in the server.

\item (25 points) Udding's solution to the starvation-free mutual
exclusion problem is reproduced below.  Explain the function of each
semaphore and counter.  Then, explain the algorithm carefully.

\begin{verbatim}
 1 P(eu); ne := ne+1; V(eu);
 2 P(qu); P(eu);	    
 3	nm := nm+1;	    
 4	ne := ne-1;	    
 5	if ne > 0 --> V(eu) 
 6	[] ne = 0 --> V(mu) 
 7	fi;		    
 8 V(qu);		    
 9 P(mu); nm := nm-1;	    
10	<cs >		    
11	if nm > 0 -->  V(mu)
12	[] nm = 0 -->  V(eu)
13	fi		    
\end{verbatim}

\end{enumerate}
\end{document}
