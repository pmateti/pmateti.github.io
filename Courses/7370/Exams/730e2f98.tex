\documentclass[12pt]{article}
\def\fat{\framebox[1mm]{\rule{0mm}{2mm}}}
\def\CSP{{\sc csp}}
\def\RPC{{\sc rpc}}
\def\SR{{\sc sr}}
\def\co{{\bf co}}
\def\oc{{\bf oc}}
\def\pa{{$\parallel$}}
\def\lb{\langle}
\def\rb{\rangle}
\def\ra{$\rightarrow$}
\def\await{{\bf await}}
\def\zimplies{\Rightarrow}
%\topmargin 0pt\oddsidemargin 17pt \evensidemargin 17pt\textheight 8.3in %\textwidth 6.0in

\parindent=0pt
\pagestyle{empty}
\begin{document}


{\bf CEG
\Large 730 Distributed Computing Principles\\[5pt]
\large Exam 2\\[5pt]
Nov 30, 1998 \quad 100 points max \quad 75 minutes\\
}

\bigskip
{\large Mateti,  Fall Quarter 1998, Wright State U}\\[-5pt]
\hrule

\begin{enumerate}

\item (8*5 points)
Explain/Discuss/Dispute/Suggest/Answer, in a {\em few} lines, the following.

\begin{enumerate}
\item Dispute:
A {\em critical assertion}, as defined in the Andrew's book, is
simply an important assertion.

\item Explain the four bold-faced terms in the following attempt at
defining what a ``safety'' property is.  Let {\sc bad} be a (i) {\bf
predicate} characterizing a ``bad'' (ii) {\bf state} of program code
segment S.  Assume that (iii) {\bf \{P\} S \{Q\}} holds.  Assume that
I is a (iv) {\bf global invariant}.  If I $\zimplies$ $\neg$ {\sc
bad}, we say that $\neg$ {\sc bad} is a safety property for S.

\item
``Some day, if not at the end of {\sc ceg} 730, you will appreciate the
book {\em Concurrent Programming} by Andrews.''  Explain if this is a
safety, or liveness property, or perhaps neither.

\item
Discuss:
\RPC\ cannot be used for peer-to-peer distributed computing.

\item
Dispute: The \RPC\ runtime locates a server for each invocation
of a server operation.

\item
Suggest a mechanism similar to RPC but in the context of Linda.


\item
Discuss:
A typical machine (node) in a distributed system can run several
concurrent processes.  Each of these processes must maintain its own
logical clock in order to correctly deal with the {\em happened
before} relation -- even if all the processes are running
on no more than two nodes.

\item
In the Andrews' distributed semaphore implementation of our textbook
that we discussed, there are numerous {\sc ack} messages being sent.
Discuss if these are unnecessary.
\end{enumerate}


\item (20 points)
Describe a distributed termination algorithm, with tokens.
The process graph {\em is} a binary tree.

\item (20+20 points) In both items below, the $a_i$ are integers, and
notationally the $a_i$ is the same as {\tt a[i]}.  Initially, the
$a_i$ sequence is sorted in the non-decreasing order, $1 \le i \le
n_t$.  In this sequence there are {\em at least} $n_d$, $0 \le n_d \le
n_t$, occurrences of an integer $d$, which need to be removed.  If the
integer $d$ occurs more than $n_d$ times, the extra occurrences
remain.  As a result, there are $n_t - n_d$ integers in the
$a$-sequence, and the indices $i$ for numbers beyond the deleted $d$'s
will need adjustment.  \\

Algorithms without thorough explanations will be considered, without
any further consideration, as deserving less than half-the-points.
High use of concurrency is expected.\\

(a) Develop a C-Linda solution for the removal of $n_d$ occurrences of
$d$.  Assume that the Linda tuple space is loaded with tuples $\lb
``nt", n_t \rb$, $\lb ``nd", n_d \rb$, $\lb ``d", d \rb$, $\lb ``p", p
\rb$ so that $\lb ``s", i, a_i \rb$, for $1 \le i \le n_t$.  At the
end, the tuple space contains $\lb ``s", i, a_i \rb$, $1 \le i \le n_t
- n_d$.  Use a fixed number $p$ of processes.  Assume that $p << n_t$.

(b) Develop a {\CSP} solution for the removal of $n_d$
occurrences of $d$.  Assume that the values $a_i$, for $1 \le i \le
n_t$, are held, one per process, by processes $P_i$, as in (but not
exactly the same as) the Small Set of Integers.  The ``user'' process
$P_0$ sends the pair $dnd(d, n_d)$ to $P_1$.  At the end, the $a_i$
sequence should remain sorted in the non-decreasing order.  

\end{enumerate}
\end{document}
