\documentstyle[12pt]{article}
\def\fat{\framebox[1mm]{\rule{0mm}{2mm}}}
\def\ellipsis{\ldots}
\def\pbar{\parallel}
\def\rar{\rightarrow}
\def\CSP{{\sc csp}}
\def\RPC{{\sc rpc}}
\def\co{{\bf co}}
\def\oc{{\bf oc}}
\def\pa{{$\parallel$}}
\def\lb{$\langle$}
\def\rb{$\rangle$}
\def\ra{$\rightarrow$}
\def\await{{\bf await}}
\def\zand{\wedge}\def\zor{\vee}	\def\znot{\neg}
%\topmargin 0pt
\parindent=0pt
\begin{document}


{\bf CEG 
\large \bf 730 Distributed Computing Principles}\\[5pt]
{Mateti,  Spring Quarter 2000, Wright State U}\\
Mid Term Exam\\
May 4, 2000 \quad 100 points max \quad 75 minutes\\
\hrule

\begin{enumerate}

\item (5*5 points)
Explain, in a few lines, the truth or falsity of the statements.

\begin{enumerate}

\item
In the context of \RPC, A procedure designed to be used remotely {\em
can} use global variables to communicate with the caller.

\item
There is no difference between safety and liveness properties.

\item
The technique of {\em passing the baton} can only be used for the
Readers/Writers Problem, and not for mutual exclusion, starvation-free
or not, of two or more processes.

\item Any group of processes communicating only via message passing
constitutes a ``distributed system.''


\item We need only have one clock per processor (not process)
in order to correctly compute a happened-before relation.

\end{enumerate}

\item (25 points) There are three \CSP\ processes $A$, $B$ and $C$
each with a local array of size $n$ of integers.  Among these $3n$
numbers, there are at least $n$ even numbers, and at least $n$ odd
numbers.  Write an algorithm in \CSP\ so that the processes send each
other integers, one at a time, eventually terminating with process $A$
having in its array only even numbers, process $C$ having in its array
only odd numbers.  You must make sure no number originally present is
lost.  You are not permitted to introduce any new arrays.  A high
degree of concurrency is expected.

\item 
\begin{enumerate}
\item (5 points)
Explain the inference rule 2.11, {\bf Concurrency Rule}, page 67, of
Andrews book reproduced below.

\begin{centering}
\{$P_i\} S_i \{Q_i\}$ are interference-free theorems, $1 \leq i \leq
n$\\[-8pt]
~.\hrulefill\\[-4pt]
$\{P_1 \zand \ellipsis \zand P_n\}$
\quad\co\quad $S_1 \pbar \ellipsis \pbar S_n$ \quad\oc\quad
$\{Q_1 \zand \ellipsis \zand Q_n\}$\par
\end{centering}

\item 
\begin{minipage}[t]{2in}
(10 points)
Determine the predicate P that characterizes the weakest deadloc-free
precondition for the program at right,
\end{minipage}\quad
\begin{minipage}[t]{2in}
\begin{tabbing}
\co\= \lb \await0\= $x \geq 3$ \=\ra0\= $x := x - 3$ \rb\kill
\co\> \lb \await \> $x \geq 3$ \>\ra \>$x := x - 3$ \rb\\
\pa\> \lb \await \> $x \geq 2$ \>\ra \>$x := x - 2$ \rb\\
\pa\> \lb \await \> $x = 1$ \>\ra \>$x := x + 5$ \rb\\
\oc
\end{tabbing}
\end{minipage}\\
\noindent that is, the largest set of states such
that, if the program is begun in a state satisfying $P$, then it will
terminate if scheduling is weakly-fair. Explain your
answer.

\item (15 points)
The following is claimed to be an implementation of
the conditional critical region {\bf region $v$ when $B$ do $S$ end}.

\begin{minipage}{2in}
{\bf
\begin{tabbing}
00\=then~\=00\=\kill
semaphore $e := 1$, $d := 0$;\\
int $nd := 0$;\\
\\
P$(e)$;\\
do not $B \rar$\+\\
  $nd := nd + 1$;\\
  V$(e)$;\quad  P$(d)$;\quad   P$(e)$\-\\
od;\\
$S$;\\
do $nd > 0 \rar$\+\\
  $nd := nd - 1$;\\
  V$(d)$\-\\
od;\\
V$(e)$
\end{tabbing}
}
\end{minipage}\quad
\begin{minipage}{2in}
(5 points) Annotate the solution with assertions.\\

(10 points) Does this solution work correctly?  If so, give a
convincing argument that it correctly implements a {\bf region}
statement.  If not, clearly explain what the error is, and give an
example of an execution sequence that causes problems.
\end{minipage}
\end{enumerate}

\item
\begin{enumerate}
\item (5 points) Discuss fully why the ``standard''
solution\quad
{\quad \tt P(m); CS; V(m);}\quad
is not starvation-free.
\item (15 points) Udding's solution to the mutual exclusion problem is
reproduced on the board.  (a) Explain the purpose of each counter variable,
and each semaphore.  State the weak semaphore assumption.  Explain the
algorithm. (b) Explain why this assumption is necessary for the
Udding's solution to be starvation-free.  (c) Simplify the Udding's
solution assuming first-come first-served queing discipline on all
semaphore operations on all three semaphores.
\end{enumerate}

\end{enumerate}
\end{document}
