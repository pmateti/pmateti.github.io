\documentstyle{article}
\def\fat{\framebox[1mm]{\rule{0mm}{2mm}}}
\def\CSP{{\sc csp}}
\def\RPC{{\sc rpc}}
\def\SR{{\sc sr}}
\def\co{{\bf co}}
\def\oc{{\bf oc}}
\def\pa{{$\parallel$}}
\def\lb{\langle}
\def\rb{\rangle}
\def\ra{$\rightarrow$}
\def\await{{\bf await}}
%\textheight=8.3in
\parindent=0pt
\pagestyle{empty}
\begin{document}


{\bf CEG 
\large \bf 730 Distributed Computing Principles\\[5pt]
\large Final Exam\\[10pt]
Dec 4, 1997 \quad 100 points max \quad 120 minutes\\
}
\bigskip
{Mateti,  Fall Quarter 1997, Wright State U}\\[-5pt]
\hrule

\begin{enumerate}

\item (6*5 points)
Explain, in a few lines, the (degree of) truth or falsity of the
following statements.

\begin{enumerate}

\item
``Some day, if not at the end of {\sc ceg} 730, you will appreciate the
book {\em Concurrent Programming} by Andrews.''  is neither a
safety, nor liveness property.

\item The weakly fair scheduling policy captures the essence
of weak semaphore assumption.

\item No \RPC\ system can guarantee that a remotely called procedure
will have the exact equivalent semantics of the corresponding
local procedure call. 

\item
The \CSP\ code {\tt [X!2 \pa\ Y!3]} is equivalent to \\
{\tt
[  true \ra\ X!2; Y!3\\
\fat\ true \ra\ Y!3; X!2\\
]}

\item
Invoking an \SR\ operation that is implemented by an {\tt in}
statement via ({\tt call} results in dynamic process creation.

\item
There is no way to implement a version of asynchronous send+receive
message passing  in Linda.

\end{enumerate}

\item (10+20 points) There are two kinds of processes, the $X$'s and the
$Y$'s, that enter a ``room''.  An $X$ process cannot leave until it
meets at least two $Y$ processes, and a $Y$ process cannot leave until
it meets at least one $X$ process.  Once a process has met the required
number of other kind of processes, that process is free to leave the
room.  Develop and explain two solutions: (i) using {\tt await}
statements only, and (ii) using semaphores only, but following the
technique of passing the baton.  Show all the details of how the
technique is applied. 

\item (20 points)
Describe a distributed termination algorithm using tokens.
The process graph is a binary tree.

\item (10+10 points) In both items below $a_i$ is the same as {\tt a[i]}
for $1 \le i \le n_t$, where $n_t$ is the size of the "array" $a$.  The
$a_i$ sequence is sorted in the non-decreasing order: $a_i \le a_{i+1}$,
$a_1$ is the smallest.  In this sequence, a specific number $d$ appears
$n_d$ or more times, $0 \le n_d \le n_t$.  We would like to remove
exactly $n_d$ occurrences of $d$.  Explain the basic ideas as well as
the algorithms.  \\

(a) Assume that the Linda tuple space is loaded with tuples $\lb ``nt'',
n_t \rb$, $\lb ``nd'', n_d \rb$, $\lb ``d'', d \rb$, $\lb ``p'', p \rb$,
and $\lb ``s'', i, a_i \rb$.  Develop a C-Linda solution for this
removal and adjustment of indices $i$ for numbers beyond the deleted
$d$'s.  At the end, the tuple space contains the $a_i$ after the removal
of $d$'s.  Use $p$ processes.  Assume that $p << n_t$.  Do not use {\tt inp},
or {\tt rdp}.

(b) Now in \CSP.  The values $a_i$, for $1 \le i \le n_t$, are now held
by processes $P_i$, one per process, as in the Small Set of Integers.
Initially, the $a_i$ sequence is sorted in the non-decreasing order.
$P_0$ sends the pair $(d, n_d)$ to $P_1$, which can send it to $P_2$,
etc.  Write a \CSP algorithm so that $n_d$ copies of $d$ are
``removed''.  At the end, the $a_i$ sequence should be sorted in the
non-decreasing order, i.e., $n_d$ additional processes at the tail end
of the process row should now be empty handed. 

\end{enumerate}
\end{document}
