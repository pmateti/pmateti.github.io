\documentstyle[12pt]{article}
\def\CSP{{\sc csp}}
\def\RPC{{\sc rpc}}
\def\fat{\framebox[1mm]{\rule{0mm}{2mm}}}
\def\co{{\bf co}}
\def\oc{{\bf oc}}
\def\pa{{$\parallel$}}
\def\lb{$\langle$}
\def\rb{$\rangle$}
\def\ra{$\rightarrow$}
\def\await{{\bf await}}
\def\rar{\rightarrow}
\def\pr{$\parallel$}

\parindent=0pt
\begin{document}

{\bf CEG 730 Distributed Computing Principles\\[5pt]
\large Mid Term Exam\\[10pt]
May 2, 1996 \quad 100 points max \quad 75 minutes\\
}
\bigskip
{Mateti,  Spring Quarter 1996, Wright State U}\\[-5pt]
\hrule

\begin{enumerate}

\item (6*5 points)
Do you agree or  disagree with the following statements?
Explain, in a few lines.

\begin{enumerate}

\item One difference between a weak semaphore {\tt w} and a strong
semaphore {\tt s} is that {\tt V(s); P(s)} is equivalent to a no-op
whereas {\tt V(w); P(w)} is not.

\item
{\sl ``A hungry philosopher will eat.''} is an example of a safety
property.

\item
The little-endian-ness of a machine has no relevance to \RPC.

\item
The \CSP\ code {\tt [X!2 \pr Y!3]} is equivalent to \par
{\tt
[  true --> X!2; Y!3 \quad \fat\ \quad true --> Y!3; X!2]
}

\item
The difference between asynchronous and synchronous message passing is
trivial: one can be implemented in terms of the other.


\item
A typical machine (node) in a distributed system can run several
concurrent processes.  Each of these processes must maintain its own
logical clock in order to correctly deal with the {\sl happened
before} relation.
\end{enumerate}


\item (30 points)
We have a \CSP\ process {\tt C} that outputs an unending stream of
non-negative numbers to a process named, say, {\tt X}.  We want {\tt
X} to be able to receive requests of the form {\tt X!nthmin(n)} from a
user process {\tt U} and respond with {\tt U!x}, where $x$ is the
$n$-th smallest number it has seen so far from {\tt C}.  If {\tt X}
has seen fewer than $n$ numbers so far, or if $n$ is greater than 10,
or less than 1, the value of $x$ to use is -1.  Write the process {\tt
X} in \CSP.  Maximize concurrency.  Process {\tt C} must not suffer
delays.


\item (20 points)
Implement the {\bf await} statements with semaphores in the following
coarse-grained {\tt deposit} and {\tt fetch} of the producer/consumer
problem, {\em using} the technique of passing the baton.  Show the
details of the derivation.

\begin{minipage}{2.5in}
\begin{verbatim}
deposit:
< await count < n ->
  buf[rear] := m;
  rear := rear mod n + 1;
  count := count + 1 >
\end{verbatim}
\end{minipage}
\begin{minipage}{2in}
\begin{verbatim}
fetch:
< await count > 0 ->
  m := buf[front];
  front := front mod n + 1;
  count := count - 1 >
\end{verbatim}
\end{minipage}

\item (10+10 points)
Study the following code as a `solution' for the starvation-free
mutual exclusion problem.  Make the same assumption regarding weak
semaphores that [Morris 79] and [Udding 86] do.

\begin{verbatim}
local x, y;  global z := 0, b := false, s[0] := 1, s[1] := 1;
x := z;
P(s[x]);
  if (z == x) then
    y := 1 - z;
    P(s[y]);
      z := y;
      repeat
        b := true;
        V(s[x]);
        P(s[x]);
      until b;
      critical section
    V(s[y])
  else
    b := false;
    critical section
  fi
V(s[x])
\end{verbatim}

(a) Annotate the solution with assertions
before and after each semaphore operation.\par

(b) Does this solution work correctly?  If so, give a
convincing argument that it correctly implements a {\bf region}
statement.  If not, clearly explain what the error is, and give an
example of an execution sequence that causes problems.

\end{enumerate}
\end{document}
