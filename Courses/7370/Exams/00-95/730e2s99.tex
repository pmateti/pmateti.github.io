\documentclass[12pt]{article}
\def\fat{\framebox[1mm]{\rule{0mm}{2mm}}}
\def\pr{$\parallel$}
\def\CSP{{\sc csp}}
\def\RPC{{\sc rpc}}
\def\co{{\bf co}}
\def\oc{{\bf oc}}
\def\pa{{$\parallel$}}
\def\lb{$\langle$}
\def\rb{$\rangle$}
\def\ra{$\rightarrow$}
\def\await{{\bf await}}
\def\SR{{\sc sr}}

\parindent=0pt
\pagestyle{empty}
\begin{document}


{\bf CEG 
\large \bf 730 Distributed Computing Principles\\[5pt]
\large Final Exam\\[10pt]
June 10, 1999 \quad 100 points max \quad 120 minutes\\
}
\bigskip
{Mateti,  Spring Quarter 1999, Wright State U}\\[-5pt]
\hrule

\begin{enumerate}
\item (5*5 points)
Explain, in a few lines, the truth or falsity of the following
statements.

\begin{enumerate}

\item  The property
``Some day, if not at the end of {\sc ceg} 730, you will appreciate
and marvel at the book {\em Concurrent Programming} by Andrews.''
is neither a safety, nor a liveness property.

\item
A {\em critical assertion} is one without which a program will abort.

\item The predicate $P \equiv (x = 6)$ characterizes the weakest
deadlock-free precondition for the program seqgment shown here.

\begin{minipage}{2.2in}
That is, the {\em largest} set of states such that, if the program is
begun in a state satisfying $P$, then it will terminate if scheduling
is weakly-fair.

\end{minipage}
\quad
\begin{minipage}{2in}
\begin{tabbing}
000\=000000000000\=\kill
\\
\co\> \lb \await\  $x \geq 3$ \>\ra $x := x - 3$ \rb\\
\pa\> \lb \await\  $x \geq 2$ \>\ra $x := x - 2$ \rb\\
\pa\> \lb \await\  $x = 1$ \>\ra $x := x + 5$ \rb\\
\oc\\
\end{tabbing}
\end{minipage}

\item
In \SR, when invoking an operation implemented by a {\tt proc} through
a {\tt send} the effect is that of a rendezvous.

\item
Java RMI is RPC applied in the context of the Java language.

\end{enumerate}

\item (10 points)
Describe a distributed termination algorithm, with tokens.
The process graph {\em is} a binary tree.

\item (20+20 points) Given: A bag $B$ of integers, the size $b$ of it,
and an integer $d$.  It is known that each element in the bag appears
exactly $d$ times.  We wish to delete the duplicates resulting in a
set of only $b/d$ distinct elements that occurred in the bag.  It is
guaranteed that $d > 0, b > 0$.

% ---
% Unfortunately, the value $d$ is unavailable and needs to be computed
% by examining the bag $B$.
% ---

As usual, we wish to maximize concurrency, we prefer a symmetric
solution, and correctness rates much much higher than efficiency.
Without a full explanation of your algorithms, you will receive no
more than 50\% of the points.

(a) Write a solution in C-Linda.  Assume that the tuple space already
contains {\tt <"B", x>} for all {\tt x} in $B$, {\tt <"d", d>} and the
size of the bag {\tt <"b", b>}.  You lose 2 points for {\em each} use
of {\tt inp} or {\tt rdp}.

(b) Write a solution in \CSP.  Process BO stores the bag $B$ in a
read-only array of $b$ integers, and has the number $d$.  But, BO is
not allowed to have any other arrays.  BO can only send one element of
$B$ at a time to any process you wish.  The process named SO should
receive distinct elements of $B$ which it should store in an array of
$b/d$ integers.  Develop the processes BO and SO.  You may use
additional processes, if you wish.

\item (10+15 points) Study the following code as a `solution' for the
starvation-free mutual exclusion problem.  Make the {\em weak
semaphore assumption}.\\

\begin{minipage}{2in}
\begin{verbatim}
semaphore s[0] := 1, s[1] := 1; /* initially */
global z := 0, b := false; /* initially */ 
local x := z, y = 0;  /* upon each entry */
P(s[x]);
  if (z == x) then
    y := 1 - z;
    P(s[y]);
      z := y;
      repeat
        b := true;
        V(s[x]);
        P(s[x]);
      until b;
      critical section
    V(s[y])
  else
    b := false;
    critical section
  fi
V(s[x])
\end{verbatim}
\end{minipage}
\quad
\begin{minipage}{2.75in}
(a) Annotate the solution with assertions
before and after each semaphore operation.\\

(b) Does this solution work correctly?  If so, give a
convincing argument that it correctly implements a {\bf region}
statement.  If not, clearly explain what the error is, and give an
example of an execution sequence that causes problems.
\end{minipage}



\end{enumerate}
\end{document}
