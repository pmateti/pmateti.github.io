\documentstyle[12pt]{article}
\def\fat{\framebox[1mm]{\rule{0mm}{2mm}}}
\def\CSP{{\sc csp}}
\def\co{{\bf co}}
\def\oc{{\bf oc}}
\def\pa{{$\parallel$}}
\def\lb{$\langle$}
\def\rb{$\rangle$}
\def\ra{$\rightarrow$}
\def\await{{\bf await}}

\parindent=0pt
\begin{document}


{\bf CEG 
\large \bf 730 Distributed Computing Principles\\[5pt]
\large Mid Term Exam\\[10pt]
May 6, 1993 \quad 100 points max \quad 75 minutes\\
}
\bigskip
{Mateti,  Spring Quarter 1993, Wright State U}\\[-5pt]
\hrule

\begin{enumerate}

\item (5*5 points)
Explain, in a few lines, the (degree of) truth or falsity of the
following statements.

\begin{enumerate}
\item
In a distributed system, there is one machine located in Cleveland,
another in Dayton.  It is not possible for these two machines to have
``shared memory.''

\item
``Every hungry philosopher will eventually get to to eat.'' is a {\sl
safety} property, not a {\sl liveness} property.

\item
The difference between asynchronous and synchronous message passing is
trivial; one can be implemented in terms of the other.


\item
In the Andrews' distributed semaphore implementation of our textbook
that we discussed, there are numerous {\sc ack} messages being sent.
Most of these are unnecessary.

\item
A typical machine (node) in a distributed system can run several
concurrent processes.  Each of these processes must maintain its own
logical clock in order to correctly deal with the {\sl happened
before} relation.

\end{enumerate}

\item (15 points)
Consider the following program segment:

\begin{tabbing}
000\=000000000000\=\kill
\co\> \lb \await $x \geq 3$ \>\ra $x := x - 3$ \rb\\
\pa\> \lb \await $x \geq 2$ \>\ra $x := x - 2$ \rb\\
\pa\> \lb \await $x = 1$ \>\ra $x := x + 5$ \rb\\
\oc\\
\end{tabbing}

Let $P$ be a predicate that characterizes the weakest deadloc-free
precondition for the program, i.e., the largest set of states such
that, if the program is begun in a state satisfying $P$, then it will
terminate if scheduling is weakly-fair.  Determine $P$.  Explain your
answer.


\item (30 points)
Given are three \CSP\ processes F, G, and H.  Each has a local array
of integers.  They may or may not be of same length, but are
definitely not empty.  All three arrays are already sorted in
non-decreasing order.  Write a \CSP\ algorithm using only these
processes so that each process determines the smallest common value.
Give key assertions in each process.  Messages should contain only one
integer at a time.


\item (15 points)
Describe a distributed termination algorithm.  Do not assume that the
process graph is a tree.

\item (5*3 points)
Udding's solution to the starvation-free mutual exclusion problem is
reproduced below.
\begin{enumerate}
\item List the initial values for all the semaphores, and counters.
\item State the weak semaphore assumption.  
\item Must each of the three semaphores satisfy this assumption?  
\item Explain what would happen if we exchange that {\tt V(eu)} of line 5
with {\tt V(qu)} of line 8.  \item Explain what would happen if we
exchange that {\tt V(eu)} of line 12 with {\tt V(qu)} of line 8.
\item Explain what would happen if we delete P(qu) from line 2 and
V(qu) from line 8.
\end{enumerate}

\begin{verbatim}
 1 P(eu); ne := ne+1; V(eu);
 2 P(qu); P(eu);	    
 3	nm := nm+1;	    
 4	ne := ne-1;	    
 5	if ne > 0 --> V(eu) 
 6	[] ne = 0 --> V(mu) 
 7	fi;		    
 8 V(qu);		    
 9 P(mu); nm := nm-1;	    
10	<cs >		    
11	if nm > 0 -->  V(mu)
12	[] nm = 0 -->  V(eu)
13	fi		    
\end{verbatim}

\end{enumerate}
\end{document}
