\documentclass[12pt]{article}
\def\pr{$\parallel$}
\def\rar{$\rightarrow$}
\def\fat{\framebox[1mm]{\rule{0mm}{2mm}}}
\def\CSP{{\sc csp}}
\def\RPC{{\sc rpc}}
\def\SR{{\sc sr}}
\def\co{{\bf co}}
\def\oc{{\bf oc}}
\def\zand{\wedge}\def\zor{\vee}	\def\znot{\neg}
\def\ellipsis{\ldots}
\def\pbar{\parallel}
\def\lb{\langle}
\def\rb{\rangle}
\def\ra{\rightarrow}
\def\await{{\bf await}}
%\textheight=8.3in
\parindent=0pt
\begin{document}
\pagestyle{empty}

{\bf CS
\large \bf 730 Distributed Computing Principles\\[5pt]
\large Exam 1\\[10pt]
Oct 24, 1996\quad 99 points max \quad 70 minutes\\
}
\bigskip
{Mateti,  Fall Quarter 1996, Wright State U}\\[-5pt]
\hrule

\begin{enumerate}

\item (8*5 points)
Explain/Discuss/Dispute/Answer, in a {\em few} lines, the following.

\begin{enumerate}
\item
Discuss the semantics of the semicolon operator as in {\tt S1 ; S2 }.

\item 
The ``standard'' solution \quad {\tt P(m); CS; V(m);} \noindent is not
starvation-free {\em only} in the distributed systems context, but is ok
when the system is a single-cpu system.

\item It is possible for a small box, say like that of a PC, to
contain a distributed system.

\item
The little-endian-ness of a machine has no relevance to \RPC.

\item
A procedure designed to be called remotely {\em must not} have
(the equivalent of Pascal's) var parameters.

\item
``The book {\em Concurrent Programming} by Andrews improves with each
reading.''  Explain if this is a safety, or liveness property, or
perhaps neither.

\item
The \CSP\ code {\tt [X!2 \pr Y!3]} is equivalent to \\
{\tt
[  true \rar\ X!2; Y!3\\
\fat\ true \rar\ Y!3; X!2\\
]}

\item
Explain the inference rule 2.11, {\bf Concurrency Rule}, page 67, of
Andrews book reproduced below.\\

\begin{centering}
\{$P_i\} S_i \{Q_i\}$ are interference-free theorems, $1 \leq i \leq
n$\\[-8pt]
~.\hrulefill\\[-4pt]
$\{P_1 \zand \ellipsis \zand P_n\}$
\quad\co\quad $S_1 \pbar \ellipsis \pbar S_n$ \quad\oc\quad
$\{Q_1 \zand \ellipsis \zand Q_n\}$\par
\end{centering}

\end{enumerate}

\newpage
\item (10 points)
Consider the following program segment:\\

{\parskip=0mm
\begin{tabular}{llllllll}
\co\    & $\lb $\await & $x$ & $\geq$ & $ 3 $ & $\ra$ & $ x := x - 3 \rb$\\
$\pbar$ & $\lb $\await & $x$ & $\geq$ & $ 2 $ & $\ra$ & $ x := x - 2 \rb$\\
$\pbar$ & $\lb $\await & $x$ & $ =  $ & $ 1 $ & $\ra$ & $ x := x + 5 \rb$\\
\oc\\
\end{tabular}
}

Let $P$ be a predicate that characterizes the weakest deadlock-free
precondition for the program, i.e., the largest set of states such
that, if the program is begun in a state satisfying $P$, then it will
terminate if scheduling is weakly-fair.  Determine $P$.  Explain your
answer.



\item (30 points)
There are two \CSP\ processes.  Process $A$ owns exactly one integer array
$a[1..m]$ and process $B$ owns exactly one integer array $b[1..n]$.
We wish $A$ to rearrange its integers of $a$ so that $a[1..i]$ are
also present in $b$, and $a[i+1..m]$ are not.  We wish $B$ to
rearrange its integers of $b$ so that $b[1..j]$ are also present in
$a$, and $b[j+1..n]$ are not.  By convention, $x[u..v]$ is considered
empty if $u > v$.

Develop a solution using \CSP.  ``Maximize'' concurrency.  Assume that
process $A$ owns $a$, and process $B$ owns $b$.  There must not be any
shared variables of whatever type.  There must not be any array
variables other than $a$ and $b$.


\item (19 points)
Using the {\em passing the baton} technique, develop a semaphore-based
implementation of conditional critical regions, whose typical syntax
is {\bf region $v$ when $B$ do $S$ end}.  Show all relevant
assertions.


\end{enumerate}
\vfill{\hfill\tiny\today}\\
\end{document}
