\documentstyle[12pt]{article}
\def\fat{\framebox[1mm]{\rule{0mm}{2mm}}}
\def\rar{\rightarrow}
\def\pr{$\parallel$}
\def\CSP{{\sc csp}}
\def\RPC{{\sc rpc}}
\parindent=0pt
\begin{document}


{\bf CEG 730 Distributed Computing Principles\\[5pt]
\large Mid Term Exam\\[10pt]
April 24, 1995 \quad 100 points max \quad 75 minutes\\
}
\bigskip
{Mateti,  Spring Quarter 1995, Wright State U}\\[-5pt]
\hrule

\begin{enumerate}

\item (30 points)
The \CSP\ solution given in Section 4.5, namely {\sl Recursive Data
Representation: Small Set of Integers}, of Hoare's \CSP\ paper is
reproduced below.  Extend the solution to respond to a command to
yield the least member of the set and to remove it from the set.

\begin{tabbing}
00\=\kill
$S(i: 1 .. 100)$ ::\+\\
$ *$\= $[\ $ \= $n$: integer; $S(i-1)$?has$(n) \rar S(0)$!false\+\\
 \fat\> $n$: integer; $S(i-1)$?insert$(n) \rar$\+\\
     $*$\= $[\ $ \= $m$: \= integer; $S(i-1)$?has$(m) \rar$\+\+\\
       $[$\ \= $m \leq n \rar S(0)!(m = n)$\\
       \fat \> $m > n \rar S(i+1)$!has$(m)$\\
       $]$ \-\-\\
  \> \fat\ $m$: integer; $S(i-1)$?insert$(m) \rar$\+\+\\
    $[\  m < n \rar S(i+1)$!insert($n$); $n := m$\\
    \fat\ $m = n \rar $ skip\\
    \fat\ $m > n \rar S(i+1)$!insert$(m)$\-\-\-\\
]\>\>]\>]\\
\end{tabbing}

\item (6*5 points)
Explain, in a few lines, the following statements.

\begin{enumerate}


\item
Given any two arbitrary processes, they are {\em concurrent}
with each other.

\item
{\sl ``A hungry philosopher will eat.''} is an example of a safety
property.

\item
``Passing the baton'' is a technique for translating {\bf await}-based
solutions into {\bf semaphore}-based solutions.

\item It is possible for a small box, say like that of a PC, to
contain a distributed system.

\item
The little-endian-ness of a machine has no relevance to \RPC.

\item
A procedure designed to be called remotely {\em must not} have
(the equivalent of Pascal's) var parameters.


\end{enumerate}

\item (10+15 points)
(a) Explain the client/server paradigm.  (b) In the context of \RPC,
describe the tasks performed by the stub procedures in the client and
in the server.


\item (15 points)
The following is claimed to be an implementation of
the conditional critical region {\bf region $v$ when $B$ do $S$ end}.

{\bf
\begin{tabbing}
00\=then~\=00\=\kill
semaphore $e := 1$, $d := 0$ initially;\\
integer $nd := 0$ initially;\\
\\
P$(e)$;\\
do not $B \rar$\+\\
  $nd := nd + 1$;\\
  V$(e)$;\quad  P$(d)$;\quad   P$(e)$\-\\
od;\\
$S$;\\
do $nd > 0 \rar$\+\\
  $nd := nd - 1$;\\
  V$(d)$\-\\
od;\\
V$(e)$
\end{tabbing}
}

(5 points) Annotate the solution with assertions.\\

(10 points) Does this solution work correctly?  If so, give a
convincing argument that it correctly implements a {\bf region}
statement.  If not, clearly explain what the error is, and give an
example of an execution sequence that causes problems.

\end{enumerate}
\end{document}
