\documentstyle[12pt]{article}
\def\CSP{{\sc csp}}
\def\RPC{{\sc rpc}}
\def\fat{\framebox[1mm]{\rule{0mm}{2mm}}}
\def\co{{\bf co}}
\def\oc{{\bf oc}}
\def\pa{{$\parallel$}}
\def\lb{$\langle$}
\def\rb{$\rangle$}
\def\ra{$\rightarrow$}
\def\await{{\bf await}}
\def\zand{\wedge}\def\zor{\vee}	\def\znot{\neg}

\parindent=0pt
\begin{document}


{\bf CEG 
\large \bf 730 Distributed Computing Principles\\[5pt]
\large Mid Term Exam\\[10pt]
Oct 23, 1995 \quad 100 points max \quad 75 minutes\\
}
\bigskip
{Mateti,  Fall Quarter 1995, Wright State U}\\[-5pt]
\hrule

\begin{enumerate}

\item (9*5 points)
Explain/Discuss, in a few lines, the (degree of) truth or falsity of
the following statements.

\begin{enumerate}
\item
Udding's solution to the starvation-free mutual exclusion problem is
reproduced below.
The algorithm would remain valid even if we introduce a new semaphore
\verb|x|, initialized to 1, and replace \verb|eu| in line 1 in both places.

\begin{verbatim}
 1 P(eu); ne := ne+1; V(eu);
 2 P(qu); P(eu);      
 3  nm := nm+1;      
 4  ne := ne-1;      
 5  if ne > 0 --> V(eu) 
 6  [] ne = 0 --> V(mu) 
 7  fi;        
 8 V(qu);        
 9 P(mu); nm := nm-1;      
10  <cs >        
11  if nm > 0 -->  V(mu)
12  [] nm = 0 -->  V(eu)
13  fi        
\end{verbatim}

\item In the above, only the \verb|mu| of
 the three semaphores ({\tt eu, qu, mu}) need satisfy the
{\sl weak semaphore} assumption.

\item The weakly fair scheduling policy is impractical.

\item It is possible for a small box, say like that of a PC, to
contain a distributed system.

\item
The difference between asynchronous and synchronous message passing is
trivial; one can be implemented in terms of the other.

\item
$(nw = 0 \zand nr \ge 0) \zor (nw = 1 \zand nr = 0)$ is the
required invariant for the Readers and Writers problem (without
priority for writers) problem.

\item

A procedure designed to be called remotely {\em must not} have
(the equivalent of Pascal's) var parameters.

\item
The little-endian-ness of a machine has no relevance to \RPC.

\item 
$P(m);$ \verb|<cs>;| $V(m)$ {\bf is} a starvation-free
solution to the critical section problem in the non-distributed case.

\end{enumerate}


\item (15 points)
Consider the following program segment:

\begin{tabbing}
000\=00000000000000\=\kill
\co\> \lb \await\quad $x \geq 4$ \>\ra\quad $x := x + 2$ \rb\\
\pa\> \lb \await\quad $x \geq 2$ \>\ra\quad $x := x + 4$ \rb\\
\pa\> \lb \await\quad $x = 9$ \>\ra\quad $x := x - 4$ \rb\\
\oc
\end{tabbing}

Let $P$ be a predicate that characterizes the weakest deadlock-free
precondition for the program, i.e., the largest set of states such
that, if the program is begun in a state satisfying $P$, then it will
terminate if scheduling is weakly-fair.  Determine $P$.  Explain your
answer.


\item (30 points)
Process $A$ has an array $a$ of $m$ integers and $B$ has an array $b$
of $n$ integers.  The two processes are allowed to send/receive only
one integer at a time.  Develop a \CSP\ algorithm for these processes
so that at the end all elements of $a$ are less than all elements of
$b$.  Assume that all the $m+n$ integers are distinct.  Assume that
the arrays $a$ and $b$ were sorted initially.  Maximize concurrency.

\item (10 points)
Describe all the major components of an \RPC\ system by
tracing a specific remote procedure call.

\end{enumerate}
\end{document}
