\documentstyle[12pt]{article}
\def\fat{\framebox[1mm]{\rule{0mm}{2mm}}}
\def\CSP{{\sc csp}}
\def\RPC{{\sc rpc}}
\def\SR{{\sc sr}}
\def\co{{\bf co}}
\def\oc{{\bf oc}}
\def\zand{\wedge}\def\zor{\vee}	\def\znot{\neg}
\def\ellipsis{\ldots}
\def\pbar{\parallel}
\def\lb{\langle}
\def\rb{\rangle}
\def\ra{\rightarrow}
\def\await{{\bf await}}
%\textheight=8.3in
\parindent=0pt
\pagestyle{empty}
\begin{document}


{\bf CEG 
\large \bf 730 Distributed Computing Principles\\[5pt]
\large Final Exam\\[10pt]
June 8, 1995 \quad 100 points max \quad 120 minutes\\
}
\bigskip
{Mateti,  Spring Quarter 1995, Wright State U}\\[-5pt]
\hrule

\begin{enumerate}

\item (10*5 points)
Explain/Discuss/Dispute/Answer, in a few lines, the following.


\begin{enumerate}
\item ``The book {\sl Concurrent Programming} by Andrews improves
with each reading.''  Explain if this is a safety, or liveness
property, or perhaps neither.

\item
Explain the inference rule 2.11, page 67, of Andrews book reproduced
below.
\begin{quote}
\begin{centering}
\{$P_i\} S_i \{Q_i\}$ are interference-free theorems, $1 \leq i \leq
n$\\[-8pt]
~\hrulefill\\[-4pt]
$\{P_1 \zand \ellipsis \zand P_n\}$
\co\quad $S_1 \pbar \ellipsis \pbar S_n$ \quad\oc
$\{Q_1 \zand \ellipsis \zand Q_n\}$\par
\end{centering}
\end{quote}
\item Describe the weakly fair scheduling policy.

\item
Consider the following program segment:

\def\await{{\bf await}}
\co\  $\lb $\await\quad $x \geq 3 \ \ra\quad x := x + 5 \rb$\par
$\pbar\  \lb $\await\quad $x \geq 8 \ \ra\quad x := x - 5 \rb$\par
$\pbar\  \lb $\await\quad $x \geq 0 \ \ra\quad x := x + 1 \rb$\par
\oc\par

Let $P$ be a predicate that characterizes the weakest deadlock-free
precondition for the program, i.e., the largest set of states such
that, if the program is begun in a state satisfying $P$, then it will
terminate if scheduling is weakly-fair.  Determine $P$.  Explain your
answer.

\item
\RPC\ cannot be used for peer-to-peer distributed computing.

\item
The \RPC\ runtime locates a server for each invocation
of a server operation.

\item
Given asynchronous message passing primitives, called {\tt asend()}
and {\tt arecv()}, build the synchronous message passing primitives
called {\tt ssend()} and {\tt srecv()}.

\item
We need only have one clock per node in order to correctly compute a
happened-before relation.

\item
There is no way to implement a ``strong'' semaphore in Linda.

\item Token-passing algorithms were discussed under the heading 
of ``asynchronous message passing.''  Does token-passing make sense
under synchronous message passing?  Why, or why not?


\end{enumerate}

\item (10+10 points)
There are two integer arrays $a[1..m]$ and $b[1..n]$.  We wish to
exchange the odd integers of $a$ with the even integers of $b$, so
that at the end $a$ has as many even integers as possible.  Obviously,
this exchange is limited by the possibly different values of $m$ and
$n$, and is dependent on how many odd numbers there are in
$a[\ellipsis]$ versus the even numbers in $b[\ellipsis]$.

Develop and explain two solutions, one using \CSP, and the other in C-Linda.  In both, maximize
concurrency.

In the \CSP\ solution,  assume that process $A$ owns $a$, and process
$B$ owns $b$.  There must not be any shared
variables whatsoever.

In the C-Linda solution, assume that the tuple space is
loaded as follows: the array $a$ is loaded as $\lb ``a", i, a[i]\rb$
for $1 \leq i \leq m$, and $b$ is loaded as $\lb ``b", j, b[j]\rb$
for $1 \leq j \leq n$.  At the end of execution of
the Linda algorithm, the tuple space should contain nothing but
$\lb ``newa", i, a[i]\rb$ tuples
for $1 \leq i \leq m$, and $\lb ``newb", j, b[j]\rb$
for $1 \leq j \leq n$ where the $a[i]$ and $b[j]$ now refer
to the current values of the conceptual arrays $a$ and $b$.
Do not use {\tt inp} or {\tt rdp}.


\item (3*10 points)
(i) What is the problem of termination detection?  (ii) Why is
termination detection a nontrivial problem in distributed systems?
(iii) Describe a token-passing distributed termination detection
algorithm.  Do not assume that the process graph is a tree.
\end{enumerate}
\end{document}
