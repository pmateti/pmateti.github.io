\documentstyle[12pt]{article}
\def\CSP{{\sc csp}}
\def\RPC{{\sc rpc}}
\def\co{{\bf co}}
\def\oc{{\bf oc}}
\def\zand{\wedge}\def\zor{\vee}	\def\znot{\neg}
\def\ellipsis{\ldots}
\def\pbar{\parallel}
\def\fat{\framebox[1mm]{\rule{0mm}{2mm}}}
\def\rar{\rightarrow}
\def\lb{\langle}
\def\rb{\rangle}
\def\ra{\rightarrow}
\def\await{{\bf await}}
\parindent=0pt
\begin{document}


{\bf CEG 
\large \bf 730 Distributed Computing Principles\\[5pt]
\large Mid Term Exam\\[10pt]
October 27, 1999 \quad 100 points max \quad 75 minutes\\
}
\bigskip
{Mateti,  Fall Quarter 1999, Wright State U}\\[-5pt]
\hrule

\begin{enumerate}

\item (5*5 points)
Explain/Discuss/Dispute/Answer, in a {\em few} lines, the following.

\begin{enumerate}
\item
As an extreme case, the piece of code $ S1 \pbar S2 $ may behave the same as
{$ S2 ; S1 $}.

\item 
The ``standard'' solution \quad {\tt P(m); CS; V(m);} \noindent is not
starvation-free {\em only} in the distributed systems context, but is ok
when the system is a single-cpu system.

\item
{\it wp}($i := 7$, $i = 8 \zor i = 7$) $=$ true.

\item
Explain the inference rule 2.11, {\bf Concurrency Rule}, page 67, of
Andrews book reproduced below.

\begin{centering}
\{$P_i\} S_i \{Q_i\}$ are interference-free theorems, $1 \leq i \leq
n$\\[-8pt]
~.\hrulefill\\[-4pt]
$\{P_1 \zand \ellipsis \zand P_n\}$
\quad\co\quad $S_1 \pbar \ellipsis \pbar S_n$ \quad\oc\quad
$\{Q_1 \zand \ellipsis \zand Q_n\}$\par
\end{centering}

\item Any group of processes communicating only via message passing
constitutes a ``distributed system.''
\end{enumerate}

\item (10 points)
Describe all the major components of an \RPC\ system by
tracing a specific remote procedure call.

\item (35 points) Consider the following as an {\tt await}-statement
based model of the classic Dining Philosophers problem.  The
expression {\tt i + 1} is modulo 5, and += is normal increment by 1.
The {\tt up} and {\tt dn} (down) refer to whether the philosopher
picked the fork up or put the fork down.

\begin{verbatim}
var up[0..4] : int := initially all 0;
var dn[0..4] : int := initially all 0;

philosopher[i: 0..4]::
  do true ->
     think
     <await up[i] == dn[i] --> up[i] += 1>
     <await up[i+1] == dn[i+1] --> up[i+1] += 1>
     eat
     <dn[i] += 1>
     <dn[i+1] += 1>
  od
\end{verbatim}

\noindent 
(a) (10 points) Using the Pass-the-Baton technique systematically
convert it to an equivalent algorithm using semaphores only.  For now,
disregard if the above is a ``solution'' or not. (b) (5 points) Is
this a ``solution'' to the Dining Philosophers problem?  If so,
explain.  If not so, fix it at the {\tt await}-statement level .  (c)
(5 points) Is this ``solution'' free from startvation?  Explain your
answers fully.  A simple yes/no will not do.


\item (30 points) The \CSP\ solution given in {\sl Recursive Data
Representation: Small Set of Integers} of Hoare's \CSP\ paper is
reproduced below.  Extend the solution to respond to a command {\it
removeMin()} that removes the least member of the set, if non-empty,
and does does nothing if the set is empty.  Explain your solution.

%\newpage
\begin{tabbing}
00\=\kill
$S(i: 1 .. 100)$ ::\+\\
$ *$\= $[\ $ \= $n$: integer; $S(i-1)$?has$(n) \rar S(0)$!false\+\\
 \fat\> $n$: integer; $S(i-1)$?insert$(n) \rar$\+\\
     $*$\= $[\ $ \= $m$: \= integer; $S(i-1)$?has$(m) \rar$\+\+\\
       $[$\ \= $m \leq n \rar S(0)!(m = n)$\\
       \fat \> $m > n \rar S(i+1)$!has$(m)$\\
       $]$ \-\-\\
  \> \fat\ $m$: integer; $S(i-1)$?insert$(m) \rar$\+\+\\
    $[\  m < n \rar S(i+1)$!insert($n$); $n := m$\\
    \fat\ $m = n \rar $ skip\\
    \fat\ $m > n \rar S(i+1)$!insert$(m)$\-\-\-\\
]\>\>]\>]\\
\end{tabbing}
\end{enumerate}
\end{document}
