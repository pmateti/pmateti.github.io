\documentstyle[12pt]{article}
\def\fat{\framebox[1mm]{\rule{0mm}{2mm}}}
\def\rar{\rightarrow}
\def\pr{$\parallel$}
\def\CSP{{\sc csp}}
\def\RPC{{\sc rpc}}
\parindent=0pt
\begin{document}


{\bf CEG 
\large \bf 730 Distributed Computing Principles\\[5pt]
\large Mid Term Exam\\[10pt]
May 7, 1998 \quad 100 points max \quad 75 minutes\\
}

\bigskip
{Mateti,  Spring Quarter 1998, Wright State U}\\[-5pt]
\hrule

\begin{enumerate}

\item (6*5 points)
Explain, in a few lines, the degree of truth of the following
statements.

\begin{enumerate}

\item
The semantics of the semicolon `;' operator as in $S_1 ; S_2$
is that execution of $S_2$ must begin immediately after the
end of $S_1$.

\item 
The ``standard'' solution \quad {\tt P(m); CS; V(m);} \quad is not
starvation-free only in the distributed systems context, but is
ok when the system is a single-cpu system.

\item
Udding's solution will not work correctly unless each of its three
semaphores satisfies the ``weak semaphore assumption.''

\item
Given any two arbitrary processes, they are {\em concurrent}
with each other.

\item
The technique of {\em passing the baton} can only be used for the
Readers/Writers Problem, and not for mutual exclusion, starvation-free
or not, of two or more processes.

\item Any group of processes communicating only via message passing
constitutes a ``distributed system.''

\item We need only have one clock per processor (not process)
in order to correctly compute a happened-before relation.

\item
In the distributed semaphore implementation, there is no need for {\sc ack}
messages.

\item
In \SR, when invoking an operation implemented by a {\tt proc} through
a {\tt send} the effect is that of a rendezvous.
\end{enumerate}

\item (30 points)
\item (4*15 points)

Process A is ready and willing to output a stream of $k$ integers, one
at a time, to process S[0].  We are interested in computing the second
and third largest integers among this stream.  Use an array of CSP
processes {\tt S[0 .. n]} to solve the problem.  Note that the
sequence is not guaranteed to have at least three distinct numbers, in
which case the second and third largest may not differ.  Choose $n$
appropriately.  As usual, we wish to maximize concurrency, and
symmetry.  Explain your solution fully.

\item (15 points)
Exercise 3.9 from Andrews is reproduced below.  Why does it not
guarantee mutual exclusion?  Does it guarantee eventual entry?  What
changes would make it correct with respect to mutual exclusion?

\begin{tabbing}
00\=00\=00\=\kill
var $lock := 0$\\
$P[i:1..n]$ ::\+\\
do true $\rar$\+\\
  $<$ await $lock = 0 >;$\\
  $lock := i$\\
  do $lock \neq i \rar$\+\\
    Delay\\
    $<$ await $lock = 0 >;$\\
    $lock := i$\-\\
  od\\
  critical section\\
  $lock := 0$\\
  non-critical section\-\\
od
\end{tabbing}


\end{enumerate}
\end{document}
