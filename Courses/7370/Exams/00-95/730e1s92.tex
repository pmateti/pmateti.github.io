\documentstyle[12pt]{article}
\def\fat{\framebox[1mm]{\rule{0mm}{2mm}}}
\def\rar{\rightarrow}
\def\pr{$\parallel$}
\def\CSP{{\sc csp}}
\def\RPC{{\sc rpc}}
\parindent=0pt
\begin{document}


{\bf CEG 730 Systems Programming (Distributed Systems) I\\[5pt]
\large Mid Term Exam\\[10pt]
May 7, 1992 \quad 100 points max \quad 75 minutes\\
}
\bigskip
{Mateti,  Spring Quarter 1992, Wright State U}\\[-5pt]
\hrule

\begin{enumerate}

\item (6*5 points)
Explain, in a few lines, the truth or falsity of the following
statements.

\begin{enumerate}

\item One difference between a weak semaphore {\tt w} and a strong
semaphore {\tt s} is that {\tt V(s); P(s)} is equivalent to a no-op
whereas {\tt V(w); P(w)} is not.

\item
Given any two arbitrary processes, they are {\em concurrent}
with each other.

\item
Any safety property can be expressed as a liveness property.

\item It is possible for a small box, say like that of a PC, to
contain a distributed system.

\item We need only have one clock per processor (not process)
in order to correctly compute a happened-before relation.

\item
In the distributed semaphore implementation, there is no need for {\sc ack}
messages.

\end{enumerate}

\item (30 points)
The \CSP\ solution given in Section 4.5, namely {\sl Recursive Data
Representation: Small Set of Integers}, of Hoare's \CSP\ paper is
reproduced below.  Extend the solution to respond to a command to
yield the least member of the set and to remove it from the set.

\newpage
\begin{tabbing}
00\=\kill
$S(i: 1 .. 100)$ ::\+\\
$ *$\= $[\ $ \= $n$: integer; $S(i-1)$?has$(n) \rar S(0)$!false\+\\
 \fat\> $n$: integer; $S(i-1)$?insert$(n) \rar$\+\\
     $*$\= $[\ $ \= $m$: \= integer; $S(i-1)$?has$(m) \rar$\+\+\\
       $[$\ \= $m \leq n \rar S(0)!(m = n)$\\
       \fat \> $m > n \rar S(i+1)$!has$(m)$\\
       $]$ \-\-\\
  \> \fat\ $m$: integer; $S(i-1)$?insert$(m) \rar$\+\+\\
    $[\  m < n \rar S(i+1)$!insert($n$); $n := m$\\
    \fat\ $m = n \rar $ skip\\
    \fat\ $m > n \rar S(i+1)$!insert$(m)$\-\-\-\\
]\>\>]\>]\\
\end{tabbing}

\item (10+15 points) (a) Discuss fully why the ``standard''
solution\\
{\tt P(m); CS; V(m);}\\
\noindent is not starvation-free.  (b) Describe
Udding's solution to the starvation-free mutual exclusion problem.  It
is not necessary to reproduce his algorithm, but you must explain
carefully the function of {\sl each} semaphore and counter used.

\item (15 points)
Lamport's protocol of Exercise 3.9 from Andrews is reproduced below.
Explain how the algorithm ensures mutual exclusion, but not eventual entry.

\begin{tabbing}
00\=00\=00\=\kill
var $lock := 0$\\
$P[i:1..n]$ ::\+\\
do true $\rar$\+\\
  $<$ await $lock = 0 >; lock := i$\\
  do $lock \neq i \rar$\+\\
    Delay\\
    $<$ await $lock = 0 >; lock := i$\\
  od\-\\
  critical section\\
  $lock := 0$\\
  non-critical section\-\\
od
\end{tabbing}


\end{enumerate}
\end{document}
