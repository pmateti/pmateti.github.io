\documentstyle[12pt]{article}
\def\fat{\framebox[1mm]{\rule{0mm}{2mm}}}
\def\pr{$\parallel$}
\def\CSP{{\sc csp}}
\def\RPC{{\sc rpc}}
\def\co{{\bf co}}
\def\oc{{\bf oc}}
\def\pa{{$\parallel$}}
\def\lb{$\langle$}
\def\rb{$\rangle$}
\def\ra{$\rightarrow$}
\def\await{{\bf await}}

\parindent=0pt
\begin{document}


{\bf CEG 
\large \bf 730 Distributed Computing Principles\\[5pt]
\large Mid Term Exam\\[10pt]
Oct 20, 1993 \quad 100 points max \quad 75 minutes\\
}
\bigskip
{Mateti,  Fall Quarter 1993, Wright State U}\\[-5pt]
\hrule

\begin{enumerate}

\item (7*5 points)
Explain/Discuss, in a few lines, the following statements.

\begin{enumerate}

\item One difference between a weak semaphore {\tt w} and a strong
semaphore {\tt s} is that {\tt V(s); P(s)} is equivalent to a no-op
whereas {\tt V(w); P(w)} is not.

\item Two processes are interference-free if they mind their
own business.

\item
Any safety property can be expressed as a liveness property.

\item It is possible for a small box, say like that of a PC, to
contain a distributed system.

\item
The difference between asynchronous and synchronous message passing is
trivial; one can be implemented in terms of the other.


\item
In the Andrews' distributed semaphore implementation of our textbook
that we discussed, there are numerous {\sc ack} messages being sent.
Most of these are unnecessary.

\item
A typical machine (node) in a distributed system can run several
concurrent processes.  Each of these processes must maintain its own
logical clock in order to correctly deal with the {\sl happened
before} relation.

\end{enumerate}


\item (15 points)
Consider the following program segment:

\begin{tabbing}
000\=000000000000\=\kill
\co\> \lb \await $x \geq 3$ \>\ra $x := x - 3$ \rb\\
\pa\> \lb \await $x \geq 2$ \>\ra $x := x - 2$ \rb\\
\pa\> \lb \await $x = 1$ \>\ra $x := x + 5$ \rb\\
\oc\\
\end{tabbing}

Let $P$ be a predicate that characterizes the weakest deadlock-free
precondition for the program, i.e., the largest set of states such
that, if the program is begun in a state satisfying $P$, then it will
terminate if scheduling is weakly-fair.  Determine $P$.  Explain your
answer.


\item (20 points)
Develop a solution in \CSP for the following problem.  Process $A$ has
a set of $m$ integers $S$ and $B$ has a set of $n$ integers $T$.  The
two processes are to exchange values one at a time until all elements
of $S$ are less than all elements of $T$.  Assume that all the $m+n$
integers are distinct.



\item (10 points) Udding's solution to the starvation-free mutual
exclusion problem is reproduced below.  Explain the function of the
semaphore {\tt qu}.

\begin{verbatim}
 1 P(eu); ne := ne+1; V(eu);
 2 P(qu); P(eu);	    
 3	nm := nm+1;	    
 4	ne := ne-1;	    
 5	if ne > 0 --> V(eu) 
 6	[] ne = 0 --> V(mu) 
 7	fi;		    
 8 V(qu);		    
 9 P(mu); nm := nm-1;	    
10	<cs >		    
11	if nm > 0 -->  V(mu)
12	[] nm = 0 -->  V(eu)
13	fi		    
\end{verbatim}




\end{enumerate}
\end{document}
