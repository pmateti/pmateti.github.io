\documentstyle[12pt]{article}
\def\fat{\framebox[1mm]{\rule{0mm}{2mm}}}
\def\pr{$\parallel$}
\def\rar{$\rightarrow$}
\def\CSP{{\sc csp}}
\def\RPC{{\sc rpc}}
\def\SR{{\sc sr}}

\textheight=9.0in
\parindent=0pt
\begin{document}
\thispagestyle{empty}

{\bf CEG 
\large \bf 730 Distributed Computing Principles\\[5pt]
\large Final Exam\\[10pt]
Nov 28, 1994\quad 100 points max \quad 120 minutes\\
}

{Mateti,  Fall Quarter 1994, Wright State U}\\[-5pt]
\hrule

\begin{enumerate}

\item (5*5 points)
Explain, in a few lines, the truth or falsity of the following
statements.

\begin{enumerate}
\item
wp$(i := 7, i = 8) = $false.

\item
Solutions to concurrent problems are starvation-prone
only in the distributed systems context, and that too
when weakly-fair scheduling is used.

\item
Interference-free implies that processes do not have shared variables.

\item
In \SR, when invoking an operation implemented by a {\tt proc} through
a {\tt send} the effect is that of a rendezvous.

\item
After a distributed termination algorithm detects termination, it
is not possible for the processes to restart on some
computation.

\end{enumerate}

\item (15 points)
Describe all the major components of an \RPC\ system by
tracing a specific remote procedure call.

\item (30 points)
Write an algorithm in C-Linda, using $np$ workers,
to compute the longest chain of numbers among a bag $B$ of
integers.  A chain is a consecutive sequence of numbers, $i, i+1, ...,
j-1, j$; the length of this chain is $j-i+1$.  Note that $B$ is not
guaranteed to have at least two distinct numbers.  As usual, we wish
to maximize concurrency.

Explain your solution fully.  Assume that the tuple space already
contains {\tt <"B", i>} for all {\tt i} in $B$, the size of the bag
{\tt <"nB", nb>}, {\tt <"min", minB>} and {\tt <"max", maxB>}.  (You
lose 5 points for each use of {\tt inp} or {\tt rdp}.)

\item (30 points)
We discussed in class the {\em Mine Sweeper} game of Windows.  Each
cell contains a small collection of processes.  Program these
in \CSP\ so that  all the cells together behave as the game does.
Explain your solution fully.
\end{enumerate}
\end{document}
