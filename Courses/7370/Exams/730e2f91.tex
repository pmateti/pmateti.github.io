\documentstyle[12pt]{article}
%\topmargin 0pt
%\oddsidemargin 17pt \evensidemargin 17pt
\textheight 8in % \textwidth 6.2in

\def\fat{\framebox[1mm]{\rule{0mm}{2mm}}}
\def\rar{\rightarrow}
\def\pr{$\parallel$}
\def\CSP{{\sc csp}}
\def\RPC{{\sc rpc}}
\def\SR{{\sc sr}}
\parindent=0pt
\begin{document}


{\bf CEG 
\large \bf 730 Distributed Systems I\\[5pt]
\large Final Exam\\[10pt]
Dec 3, 1991 \quad 100 points max \quad 120 minutes\\
}
\bigskip
{Mateti,  Fall Quarter 1991, Wright State U}\\[-5pt]
\hrule

\begin{enumerate}

\item (6*5 points)
Explain, in a few lines, the truth or falsity of the following
statements.

\begin{enumerate}

\item
The Byzantine Generals Problem can always be solved if the total
number of generals is 2 regardless of whether the number of traitors
is 0, 1 or 2.

\item
In \SR, when invoking an operation implemented by a {\tt proc} through
a {\tt send} the effect is that of a rendezvous.

\item
{\sl ``A hungry philosopher will eat.''} is an example of a safety
property.

\item
Semaphores are not heavily used in most {\em parallel} applications as
opposed to most {\em concurrent} systems.

\item
If a client and server happen to reside on the same machine the
results will be unpredictably erroneous.

\item
Semaphores, weak or strong, cannot be implemented in Linda.


\end{enumerate}

\vfill
\item (5+10 points)\\
(a) What does ``distributed termination'' mean?\\
(b) Describe the Dijkstra-Sch\"olten distributed termination algorithm.

\vfill
\item (20 points)
Write an algorithm in Linda for the following problem.  Given, in the
tuple space {\sc ts}, a bag $B$ of integers {\tt <"B", i>} for all
{\tt i} in $B$, and the size of the bag {\tt <"nB", nb>}, compute the
largest number {\tt <"maxB", mx>} in the bag.

\vfill
\item (20 points)
The \CSP\ solution given in Section 4.5, namely {\sl Recursive Data
Representation: Small Set of Integers}, of Hoare's \CSP\ paper is
reproduced below.  Extend the solution to respond a command to
yield the least member of the set and to remove it from the set.

\newpage
\begin{tabbing}
00\=\kill
$S(i: 1 .. 100)$ ::\+\\
$ *$\= $[\ $ \= $n$: integer; $S(i-1)$?has$(n) \rar S(0)$!false\+\\
 \fat\> $n$: integer; $S(i-1)$?insert$(n) \rar$\+\\
     $*$\= $[\ $ \= $m$: \= integer; $S(i-1)$?has$(m) \rar$\+\+\\
       $[$\ \= $m \leq n \rar S(0)!(m = n)$\\
       \fat \> $m > n \rar S(i+1)$!has$(m)$\\
       $]$ \-\-\\
  \> \fat\ $m$: integer; $S(i-1)$?insert$(m) \rar$\+\+\\
    $[\  m < n \rar S(i+1)$!insert($n$); $n := m$\\
    \fat\ $m = n \rar $ skip\\
    \fat\ $m > n \rar S(i+1)$!insert$(m)$\-\-\-\\
]\>\>]\>]\\
\end{tabbing}


\item (15 points)
Study the following code as a `solution' for the starvation-free
mutual exclusion problem.  Make the same assumption regarding weak
semaphores that [Morris 79] and [Udding 86] do.

\begin{verbatim}
local x, y;  global z=0, b=false, s[0]=1, s[1]=1;
x := z;
P(s[x]);
  if z = x then
    y := 1 - z;
    P(s[y]);
      z := y;
      repeat
        b := true;
        V(s[x]);
        P(s[x]);
      until b;
      <critical section>
    V(s[y])
  else
    b := false;
    <critical section>
  fi
V(s[x])
\end{verbatim}
Either give a convincing argument that the above algorithm is correct,
or give a scenario demonstrating a situation that violates the
requirements of the problem.
\end{enumerate}
\end{document}



