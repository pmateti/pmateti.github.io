\documentclass[12pt]{article}
\textheight=8in
\newcommand{\fat}{\framebox[1mm]{\rule{0mm}{2mm}}}
\newcommand{\CSP}{CSP}
\newcommand{\RPC}{RPC}
\newcommand{\co}{{\bf co}}
\newcommand{\oc}{{\bf oc}}
\newcommand{\pa}{{$\parallel$}}
\newcommand{\lb}{$\langle$}
\newcommand{\rb}{$\rangle$}
\newcommand{\ra}{$\rightarrow$}
\newcommand{\await}{{\bf await}}
\newcommand{\zand}{\wedge}
\newcommand{\zor}{\vee}
\newcommand{\znot}{\neg}

\parindent=0pt
\begin{document}


{\bf CEG 
\large \bf 730 Distributed Computing Principles\\[5pt]
\large Mid Term Exam\\[10pt]
Oct 1997 \quad 100 points max \quad 75 minutes\\
}
\bigskip
{Mateti,  Fall Quarter 1997, Wright State U}\\[-5pt]
\hrule

\begin{enumerate}

\item (6*5 points)

Explain/Discuss/Dispute/Answer, in a {\em few} lines, the following.

\begin{enumerate}

\item
Discuss the semantics of the fat-bar operator as in {\tt S1 \fat\ S2 }.

\item
The little-endian-ness of a machine has no relevance to \RPC.

\item
What does \verb|svc_run()| do?  Why does it not ``return''?

\item
Explain what {\em dispatching} is, in the context of \RPC.

\item
If a client and server happen to reside on the same machine the
results will be unpredictably erroneous.

\item
``Every philosopher will eventually get hungry.''  Explain
if this is a safety, or liveness property, or perhaps neither.
\end{enumerate}


\item (10+5+10 points)
\begin{enumerate}
\item
Explain the difference(s) in meaning of the notation
$ \{ P \} \quad S \quad \{ Q \} $
versus
$ wp(S, Q) $.

\item
Explain the {\bf Concurrency Rule}, page 67, inference rule 2.11, of
Andrew's book reproduced below.\\

\begin{centering}
\{$P_i\} S_i \{Q_i\}$ are interference-free theorems, $1 \leq i \leq
n$\\[-8pt]
~.\hrulefill\\[-4pt]
$\{P_1 \zand \ldots \zand P_n\}$
\quad\co\quad $S_1 \parallel \ldots \parallel S_n$ \quad\oc\quad
$\{Q_1 \zand \ldots \zand Q_n\}$\par
\end{centering}

\item
Consider the following program segment:

\begin{tabbing}
\co\= \lb \await0\= $x \geq 3$ \=\ra0\= $x := x - 3$ \rb\kill
\co\> \lb \await \> $x \geq 3$ \>\ra \>$x := x - 3$ \rb\\
\pa\> \lb \await \> $x \geq 2$ \>\ra \>$x := x - 2$ \rb\\
\pa\> \lb \await \> $x = 1$ \>\ra \>$x := x + 5$ \rb\\
\oc
\end{tabbing}

Let $P$ be a predicate that characterizes the weakest deadloc-free
precondition for the program, i.e., the largest set of states such
that, if the program is begun in a state satisfying $P$, then it will
terminate if scheduling is weakly-fair.  Determine $P$.  Explain your
answer.
\end{enumerate}


\item (5*5 points)
The solutions by Udding and Morris to the starvation-free mutual exclusion problem are reproduced below.
\begin{enumerate}
\item State the weak semaphore assumption.
\item What are the implications of this assumption with respect to
the code fragment {\tt V(s); P(s)} ? 
\item What are the initial values of the six semaphores below?
\item
Must each of the three semaphores in each solution satisfy this assumption?
\item Is the functionality/purpose of the {\tt q}-semaphores the same? 
\end{enumerate}

\begin{verbatim}

Udding's solution               | Morris' solution

 1 P(eu); ne := ne+1; V(eu);    | P(em); ne := ne+1; V(em);
 2 P(qu); P(eu);                | P(qm);
 3  nm := nm+1;                 |  nm := nm+1;             
 4  ne := ne-1;                 |  P(em); ne := ne-1;
 5  if ne > 0 --> V(eu)         |  if ne > 0 --> V(em);V(qm)
 6  [] ne = 0 --> V(mu)         |  [] ne = 0 --> V(em);V(mm)
 7  fi;                         |  fi;
 8 V(qu);                       |
 9 P(mu); nm := nm-1;           | P(mm); nm := nm-1;             
10  <cs >                       |  <cs >                   
11  if nm > 0 -->  V(mu)        |  if nm > 0 -->  V(mm) 
12  [] nm = 0 -->  V(eu)        |  [] nm = 0 -->  V(qm) 
13  fi                          |  fi
\end{verbatim}

\item (20 points)
Process $A$ has an array $a$ of $1..m$ integers and $B$ has an array $b$
of $1..n$ integers.  The two processes are allowed to send/receive only
one integer at a time.  Develop a \CSP\ algorithm for these processes
so that at the end $a[1]$ is the smallest of all elements of $a$ and
$b$.  While elements may have got shuffled, at the end the two arrays
together must have the same bag of elements that they did at the
beginning.  Do not assume that the $m+n$ integers are distinct.
Do not assume that the arrays $a$ and $b$ were sorted initially.
Maximize concurrency.  Use more processes if you wish.

\end{enumerate}
\end{document}
