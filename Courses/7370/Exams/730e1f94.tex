\documentstyle[12pt]{article}
\def\fat{\framebox[1mm]{\rule{0mm}{2mm}}}
\def\CSP{{\sc csp}}
\def\RPC{{\sc rpc}}
\parindent=0pt
\begin{document}


{\bf CEG 
\large \bf 730 Distributed Computing Principles\\[5pt]
\large Mid Term Exam\\[10pt]
Oct 19, 1994 \quad 100 points max \quad 75 minutes\\
}
\bigskip
{Mateti,  Fall Quarter 1994, Wright State U}\\[-5pt]
\hrule

\begin{enumerate}

\item (5*5 points)
Explain, in a few lines, the truth or falsity of the following
statements.

\begin{enumerate}
\item There are some existing programs that cannot be rewritten
to take advantage of distributed systems.

\item
Every distributed system must be geographically
distributed.

\item
Often there is no difference between safety and liveness properties.

\item Asynchronous and synchronous message passing are fundamentally
different.

\item
We need only have one clock per node in order to correctly compute a
happened-before relation.



\end{enumerate}

\item (30 points)
We have a \CSP\ process {\tt C} that outputs an unending stream of
non-negative numbers to a process named, say, {\tt X}.  We want {\tt
X} to be able to receive requests of the form {\tt X!nthmax(n)} from a
user process {\tt U} and respond with {\tt U!x}, where $x$ is the
$n$-th largest number it has seen so far from {\tt C}.  If {\tt X}
has seen fewer than $n$ numbers so far, or if $n$ is greater than 10,
or less than 1, the value of $x$ to use is -1.  Write the process {\tt
X} in \CSP.  Maximize concurrency.  Process {\tt C} must not suffer
delays.

\item (20 points)
Describe the main ideas of the distributed semaphore algorithm.


\item (10+15 points)  Udding's solution to the mutual exclusion problem
is reproduced below.  (a) Explain the purpose of each of the
semaphores, using invariants. (b)  Explain what could happen if
we delete {\tt P(qu)} and {\tt V(qu)} operations.

\begin{verbatim}
 1 P(eu); ne := ne+1; V(eu);
 2 P(qu); P(eu);        
 3    nm := nm+1;        
 4    ne := ne-1;        
 5    if ne > 0 --> V(eu) 
 6    [] ne = 0 --> V(mu) 
 7    fi;            
 8 V(qu);            
 9 P(mu); nm := nm-1;        
10    <cs >            
11    if nm > 0 -->  V(mu)
12    [] nm = 0 -->  V(eu)
13    fi            
\end{verbatim}

\end{enumerate}
\end{document}
