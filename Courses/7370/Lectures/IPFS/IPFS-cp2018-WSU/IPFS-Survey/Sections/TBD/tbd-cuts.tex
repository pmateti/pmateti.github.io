
This is a collection of cuts made.  Just for archival purpose.

Sections dropped or merged elsewhere: challenges.tex, problem.tex.


Mobile phones are now indispensable devices. We use our smart phones
for all our day to day works. Among all the mobile operating systems,
Android is the most widely used.

% \subsection{Smartphones in the Hands of Criminals}

%\nocite{almohri2014}
\nocite{burguera2011crowdroid}
\nocite{gehani2012spade}
\nocite{Nikhil2015}
\nocite{elenkov2014android}
\nocite{isohara2011kernel}
\nocite{Mateti}
\nocite{kwong2012droidscope}
\nocite{lin2013identifying}
\nocite{linuxkernelarchives}
\nocite{connectivitymanager}
\nocite{linuxprocesses}
\nocite{SystemCalls}
\nocite{straceinandroid}
\nocite{dropboxcloud}
\nocite{wifimanager}
\nocite{yoon2012appscope}
\nocite{armando2014mobile}
\nocite{connectivitymanager}
\nocite{wifimanager}
\nocite{dropboxcloud}

  Malware harms the mobile phones like draining the battery,
to more seriously stealing the user's data, to making the device be a
part of attacks.

\subsection{Android OS}

Android is (largely) an open source mobile operating system.
Following are the layers\footnote{\url{https://source.android.com/}}
found above the Linux kernel.

\subsection{What is a Process?}

A process is a program in execution and is meant to carry out tasks
within the operating system.  A process is a run time volatile entity
created by the system call {\tt exec()}.  Every running app is at
least one process, and many processes contain several threads.
Android includes several Linux level command line utilities such as
{\tt ps, top, strace} and {\tt ltrace}.

\begin{itemize}
\item Hardware Abstraction Layer (HAL) - It is resposible for providing
  interface that show the hardware capabilities to the above Java API
  framework.
\item Android Runtime (ART) - ART is the equivalent of Java Virtual
  Machine.  Each app has its own instance of ART and runs in its own
  process.  ART provides the following features:
  ahead of time and just in time compilation,
  garbage collection, and
  debugging support.

\item Native C/C++ Libraries - Many Android system components are
  built using written C and C++.
\item Java API Framework - Apps are generally, but not always, written
  in Java using the Android Java API framework.  They provide a rich
  feature set and includes the following:
  View System,
  Resource Manager,
  Notification Manager,
  Activity Manager, and
  Content Providers

\item System Apps - Android has a set of system apps (located in {\tt
  /system/app} for various functionalities like e-mail, calendar,
  internet browsing, SMS and more.  The system apps cater to the end
  users and also provide key capabilities for third party apps.
\end{itemize}



We use Android's API such as {\tt RunningAppProcessInfo, Activity
  Manager}, etc., to retrieve the details of present running
processes.  


If there is a system that
monitors each application by analyzing the processes of the
application we will be able to track the activities within the Android
OS.  Be it a useful or a malicious application, it would be of great
help to know what these are doing inside the system when something
goes wrong.


All components of an application run in the same process.  But if we
want to change which process a certain component belongs to, we can
change this in the manifest file.

In the end,
{\tt file strace} should show {\tt strace: ELF 32-bit LSB executable,
  ARM, version 1 (SYSV), for GNU/Linux 2.6.14, statically linked}.

Get
an instance of this class by calling
\begin{lstlisting}[breaklines=true]
Context.getSystemService(Context.CONNECTIVITY_SERVICE).
\end{lstlisting}
The primary responsibilities of this class are to

\begin{itemize}

 	\item    Monitor network connections (Wi-Fi, GPRS, UMTS, etc.).
    \item Send broadcast intents when network connectivity changes.
    \item Attempt to fail over to another network when connectivity to a network is lost.
    \item Provide an API that allows applications to query the coarse-grained or fine-grained state of the available networks.
    \item Provide an API that allows applications to request and select networks for their data traffic.
\end{itemize}
We use the method  getActiveNetwork() of this class to the network connection details like Network Type, Roaming State and Extra Info.

Returns a Network object corresponding to the currently active default data network.  

calling
\begin{lstlisting}


         Context.getSystemService(Context.WIFI_SERVICE).  
         \end{lstlisting}
         It deals with several categories of items.
\begin{itemize}

	\item The list of configured networks.  
	\item The list can be viewed and updated, and attributes of individual entries can be modified.
    \item The currently active Wi-Fi network, if any.  
    \item Connectivity can be established or torn down, and dynamic information about the state of the network can be queried.
    \item Results of access point scans, containing enough information to make decisions about what access point to connect to.
    \item It defines the names of various Intent actions that are broadcast upon any sort of change in Wi-Fi state.  
    \end{itemize}


  \begin{minipage}{0.24\textwidth}
    \includegraphics[width=\textwidth]{Figures/application_details}
    \caption{Application Details}
    \label{applicationdetails}
  \end{minipage}

\begin{figure}[hb]\centering
  \begin{minipage}{0.24\textwidth}
    \includegraphics[width=\textwidth]{Figures/past_process_list}
    \caption{List of Past Process}
    \label{pastprocess}
  \end{minipage}\hfill
  \begin{minipage}{0.24\textwidth}
    \includegraphics[width=\textwidth]{Figures/past}
    \caption{Details of the past process mv}
    \label{pastprocess}
  \end{minipage}
\end{figure}

\subsection{Battery Information}\label{seBattery}

\TBD{Not part of this article.}
The {\tt BatteryManager} class contains strings and constants used for
values in the ACTION BATTERY CHANGED Intent.  We discover the level of
battery, the health of the battery, whether it is Good, Overheated,
Dead, Over Voltage, Failure, the status of the battery (Charging,
Discharging, Full, Not Charging) as follows.

{\small
\begin{lstlisting}
  int level = intent.getIntExtra("level", 0); 
  int scale = intent.getIntExtra("scale", 100); 
  String lStr = String.valueOf( level * 100 / scale ) + '%';
  data.add( new String[]{
    getString( R.string.batt_level ), lStr} );
  int health = intent.getIntExtra("health", BatteryManager.BATTERY_HEALTH_UNKNOWN );
  int status = intent.getIntExtra("status", BatteryManager.BATTERY_STATUS_UNKNOWN );
  int temperature = intent.getIntExtra("temperature", 0 ); 
  int voltage = intent.getIntExtra("voltage", 0 ); //$NON-NLS-1$
  String vStr = String.valueOf( voltage ) + "mV"; //$NON-NLS-1$
  data.add( new String[]{
    getString( R.string.batt_voltage ), vStr} );
\end{lstlisting}}

\hrule

title={{AppScope}: Application Energy Metering Framework for Android
  Smartphone Using Kernel Activity Monitoring.}
\citep{yoon2012appscope}

\subsection{{\tt strace} of Running Process}

\TBD{Delete} 
When a process terminates, strace saves its logged data to a text
file.  Our current implementation did not consider the strace output
of currently running process.  The strace log files of currently
running process should also be read and interpreted, by using
buffering methods, to get the provenance of currently running process.
This will make this work, more resourceful.

\subsection{Disadvantages Of Using JNI}
\begin{itemize}
	\item Portability issues because of using native codes.
	\item Core dumps and segmentation faults may rise due to poor coding practice, which in turn may cause the whole application to crash.  
	\item Debugging runtime error in native code is very difficult.  
	\item Potential security risk.
\end{itemize}

All the logged details will be uploaded in cloud server.  A log
uploader will be uploading all the logs to clouds in a configurable
way like configurable time interval for log upload.  All the uploaded
logs will be deleted from the device.

\subsection{Use of Provenance in Security}

title={Identifying {Android} Malicious Repackaged Applications by
                  Thread-Grained System Call Sequences}
\citep{lin2013identifying}

