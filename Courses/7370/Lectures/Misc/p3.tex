\documentstyle[12pt]{article}
\newcounter{iex}\setcounter{iex}{1}
\def\itm{\medskip\noindent \theiex. ~\stepcounter{iex}}
\def\CSP{{\sc csp}}
\def\SR{{\sc sr}}
\def\ISIS{{\sc isis}}
\parindent=0pt
\parskip=0pt
\begin{document}


{\bf CS
\Large \bf 7370 Distributed Computing Principles}\\[5pt]
\bigskip

{Fall Qtr 1995, Wright State U}

\bigskip

\section*{Project 3}

Project 1 was about program-comprehension.  The {\sl WB} is a
multi-workstation oriented {\sl WhiteBoard} client-server program, and
deals with {\sc rpc}-usage and {\sc x11} through a pre-packaged module
of mine.  All relevant source and documentation is in
\verb|/usr/local/lib/Languages/7370/| 
Your task in Project 1 was to run the server on one node, and at least
four clients on four different nodes of at least three different
architectures and two different whiteboards.


Project 3 asks you to modify the code of Project 1 to make it better.

\begin{enumerate}

\item Server responds to query from a client with full information
(name of the board, names of the host machines where the boards are in
use, etc.) regarding all whiteboards.

\item Server responds to a request to clone itself on a named node.
The old server(s) together with the newly created clone together
serve the pool of client whiteboards by dividing the clients.

\item A server responds to a request to transfer a named client
whiteboard
from one server to another.

\end{enumerate}


{\vfill\tiny Oct 1995}
\end{document}
