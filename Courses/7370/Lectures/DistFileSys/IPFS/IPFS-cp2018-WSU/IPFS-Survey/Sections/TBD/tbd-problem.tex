\section{Problem Solution Architecture}\label{problem}
This section describes solutions to various problems that evolved while developing and implementing the thesis.

\subsection{Strace in Android}
Strace is not a built in tool in Android. So we have to install it in Android. The Android device must be rooted for installing strace. The following steps explains how to install strace in Android.
\begin{itemize}
	\item The strace source distribution is available in SourceForge. Download and extract it in the linux machine.
	\item The compiler information is set as follows:
	\begin{lstlisting}
	export CC=your_cross_compiler_folder/bin/arm-none-linux-gnueabi-gcc
	export STRIP=your_cross_compiler_folder/bin/arm-none-linux-gnueabi-strip
	export CFLAGS="-O2 -static"
	\end{lstlisting}
	
	\item Add the bin folder within the cross\_compiler folder to the \$PATH.
	\item Move to the strace folder (where the source is extracted), and execute the following command.
	
	\begin{lstlisting}
	./configure --host=arm-linux
	\end{lstlisting}
	
	\item Compile
	\begin{lstlisting}
	make
	\end{lstlisting}
	\item Verify result
	\begin{lstlisting}
	file strace
	\end{lstlisting}
	The output would be like
	\begin{lstlisting}
	strace: ELF 32-bit LSB executable, ARM, version 1 (SYSV), for GNU/Linux 2.6.14, statically linked, not stripped
	\end{lstlisting}
\end{itemize}

\subsection{Filtering Strace Output}
The output of
strace
can quickly become very large and synchronous output to
strerr
or a log file may
greatly impair program performance. If we know up front which system calls are of importance, we
may limit the output to one or a few kernel calls (or e.g. with
-e file
system calls with file names). This
slows down program execution considerably less and makes subsequent evaluation of the
strace	output	much easier. For our thesis, we concentrate mostly on file and network related system calls.

\subsection{Disadvantages Of Using JNI}
\begin{itemize}
	\item Bad C/C++ code in native library will cause core dumps / segmentation faults that the JVM cannot recover from. This may result in crashing of the whole app.
	\item Difficult to debug runtime error in native code.
	\item Potential security risk.
\end{itemize}

\subsection{Storage}
A cloud storage for the collected log files is preferred, as the log files keeps on getting accumulated while the processes are running.

All the logged details will be uploaded in cloud server. A log uploader will be uploading all the logs to clouds in a configurable way like configurable time interval for log upload. All the uploaded logs will be deleted from the device.

JSON and OAuth API is used for cloud integration. There are many free cloud storage providers available. We can choose any one of them, for our cloud upload.
