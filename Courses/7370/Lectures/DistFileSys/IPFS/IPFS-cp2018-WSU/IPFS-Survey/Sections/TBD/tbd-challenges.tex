\section{Challenges of Android Dashboard of Processes}\label{Challenges}

This section describes solutions to various problems that evolved
while developing and implementing the paper.

\subsection{Filtering Strace Output}

The output of strace can be enormous.  It analyses each and every
system calls executed by the running process.  We can reduce the size
of the strace output by limiting the output to one or a few system
calls that are of most important for our work.  This can be achieved
by usibg the option {\tt -e} along with system calls we want to
analyze: e.g., strace -e open PID, analyzes the open system call for
the given process id.  In this paper, we concentrate mostly on file
and network related system calls.  To analyze all the file related
system calls, we can use {\tt strace -e trace=file PID} and for all
network related system calls we can use {\tt strace -e trace=network
  PID}.
	
\subsection{Disadvantages Of Using JNI}
\begin{itemize}
	\item Portability issues because of using native codes.
	\item Core dumps and segmentation faults may rise due to poor coding practice, which in turn may cause the whole application to crash.  
	\item Debugging runtime error in native code is very difficult.  
	\item Potential security risk.
\end{itemize}

\subsection{Storage}

A cloud storage for the collected log files is preferred, as the log
files keeps on getting accumulated while the processes are running.

All the logged details will be uploaded in cloud server.  A log
uploader will be uploading all the logs to clouds in a configurable
way like configurable time interval for log upload.  All the uploaded
logs will be deleted from the device.

JSON and OAuth API is used for cloud integration.  There are many free
cloud storage providers available.  We can choose any one of them, for
our cloud upload.
