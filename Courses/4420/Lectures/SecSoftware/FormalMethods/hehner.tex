\documentclass[12pt]{article}
\parskip=5pt
\textheight=8.5in% so that this prints as 2 pages
\begin{document}
\pagestyle{empty}

\subsection*{A Few Opinions of Professor Hehner}
{\small Extracted from the book {\sl The Logic of
Programming}, Prentice-Hall 1984, by Eric Hehner, Professor of Computer
Science, University of Toronto.}


\subsubsection*{Professional Responsibility}

A typical guarantee on anything but software is something like
this: ``If there are defects in the material or construction of this
product, it will be promptly replaced free of charge, or all money
will be cheerfully refunded.''  The corresponding guarantee for the
software would be: ``If there are errors in the syntax or semantics of
this program, it will be promptly corrected free-of-charge, or all
money will be cheerfully refunded.''  How very different is that from
the typical software ``guarantee'', which is: ``If there are errors in
this program, we may, if we choose, try to correct them, for an
additional charge.  Under no circumstances will money be refunded.''

Physical objects (bridges, buildings, bicycles) require
maintenance to keep them clean, working, and safe; periodically, worn
and broken parts need to be replaced.  Programs do not get dirty, wear
out, or break; if ever a program is correct, it remains correct.  In
short, programs do not need maintenance.  Yet there are two distinct
activities that are commonly called program maintenance.  The first is
the one mentioned in the ``typical software guarantee'';
``maintenance'' is used as a euphemism for ``correction''.  This
allows a programmer who is incapable of writing a correct program to
declare success with an incorrect program, and leave the problem to
someone else (for more money).  This use of the term is just plain
dishonest.

The other activity that is commonly called program maintenance
is program modification to meet a change in specifications.  After a
program is in use, its users sometimes want it to do something
different, usually something more, than originally planned.  This is
fair enough; users do not always know at first what they want or need.
And it is fair enough to ask more money for more programming.  But
this activity should no more be called program maintenance than adding
a new wing to a building should be called building maintenance.
Reprogramming, as we may more properly call it, is not a separate
subject; there is no special way to make changes and additions to a
program that is different from programming in the first place.  Of
course, it will be easier if the original program is well-written and
well-documented, and we should always keep that in mind.

If a bridge collapses due the incompetence of an engineer, the
guilty engineer can be sued and jailed.  Doctors, lawyers, and people
of any profession affecting the well-being of people, are liable to
malpractice charges if they fail to be properly educated in the their
profession, or are presently many programmers whose poor efforts can
aptly be described as malpractice.  Improperly trained and using
grotesque programming languages, they write programs upon which our
lives and our fortunes frequently depend.  For shame, we can and must
do much better.  There is no good excuse for continuing these present
poor practices.


\subsubsection*{On Mathematics}

A student who wished to learn computer programming asked me 
for advice on which of two courses to take.  One was more mathematical
than the other, and he did not like Mathematics.  Since he was a pianist,
I asked him which course one should take to become a composer, if one 
does not like music.

\subsubsection*{On Programming Books and Courses}

Any book whose title or preface suggests that it will make 
programming easy is a fraud.  Any book that avoids the "difficult"
topics of formal semantics and analysis of efficiency avoids teaching
programming, and should itself be avoided.  Such books actually make 
programming more difficult by denying the would-be programmer the 
necessary intellectual tools.  Avoiding difficult topics may serve 
universities whose goal is to maximize enrollment, but it is to the
detriment of students who cannot ultimately succeed to delay the 
realization that programming is difficult for them, and it is unfair
to good students to delay their education and the enjoyment of mastering
the subject.  We have plenty of poor programmers now; we need a few good
ones.

	But the unfortunate trend in programming courses has been to 
encourage the attitude "try it and see if it works" as a method of 
program composition.  [...] Assignments seem to demand a running program
(with no apparent errors) rather than a correct one (apparent that there
are no errors).

\subsubsection*{On Being Logical}

This book is dedicated to my son Joshua, who has shown me that people
begin logical, and are urged and pressed to conform to an illogical and
inconsistent world; may he continue to resist.


\end{document}
