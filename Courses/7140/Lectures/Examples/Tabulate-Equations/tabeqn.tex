\documentclass[12pt]{article}
\topmargin=0pt
\parskip=10pt
\parindent=0pt

\newfont{\st}{cmsltt10 scaled \magstep1} 

\def\bnf{\st}
\def\zdef{\bf\ \widehat{=}\ }
\def\vbar{{\tt |}}
\def\forall{{\bf forall }}
\def\forsome{{\bf forsome }}
\def\ffi{{\tt fi}}
\def\ffo{{\tt fo}}

\begin{document}

{\centering
{\Large Two Specifications of Tabulated Equations\\[5pt]}

Prabhaker Mateti\\
}

\bigskip

{\small This note contains both a ``good''  answer, and
a ``really''\footnote{Do you really know the meaning of ``really''?
Look it up in a decent dictionary, such as Longman Dictionary
of Contemporary English.}
precise answer to an exam question in a
Software Engineering class.  The occasional commentary by this
instructor is set in the type size of this paragraph.  }

\section{Informal Statement of the Problem}

Consider the informal description, given in the next paragraph, of a
program.  Give a carefully written specification of it, using
well-chosen notations, and words.  Pretend that the requirements
analysis is already done: That is, wherever relevant in the specs,
explain in English the appropriate parts of requirements.

{\sl
\parindent=0pt
Our mathematician friend wants a program that can ``tabulate'' a
sequence of symbolic equations found in typical linear algebra into a
matrix form.

For example, given the input shown in Figure \ref{inp},

\begin{figure}[hb]
\begin{verbatim}
3x + 5y + 714z = 12g + 53b,
76489x +3z = 2g+8b+99a + 1024,
           x+ 49 = 7g,
1234567y+ 2z = a + 999b + 6x + 911,
 + 3x + 5y =  53b + 12g.
\end{verbatim}
\caption{The Input}
\label{inp}
\end{figure}


\begin{figure}
\begin{verbatim}
    3x +       5y + 714z      = 12g +  53b,
76489x +              3z      =  2g +   8b + 99a +      1024,
     x +                   49 =  7g,
         1234567y +   2z      =       999b +   a + 6x +  911,
    3x +       5y             = 12g +  53b.
\end{verbatim}
\caption{Output Corresponding to Input of Figure \protect{\ref{inp}} }
\label{out}
\end{figure}

your program should produce the result shown in Figure \ref{out}.


\noindent  Our friend  tells us that her equations do not contain
minus signs, nor floating point numbers, but that sometimes the
coefficients are larger than can fit in 64-bits, and the equations
become so long that she has to write them on a long scroll of paper
sideways.  We impressed her by remarking that we will imagine the
paper width to be infinity.
}

\section{A Good Solution}

{\small There are essentially three things we must consider. (1) What
are the preconditions? (2) Output looks good.  (3) Output is
equivalent to the given input.  This solution is deliberately written
in an informal way to emphasize what needs to be specified versus how
to specify it.  So, find all the ambiguities -- but, do not just nit
pick! }

Since nothing particular is said about how i/o ought to be, we assume
that the required {\tt tabulateeqn} program takes its textual input
from {\tt stdin} and outputs to {\tt stdout}.


\subsection{Syntax of Equations}

The input must be a sentence derived from the non-terminal {\bnf eqns}
of grammar $G$ given below.  Note that (a) an equation can span
multiple lines, and (b) terms of the same variable may occur multiply
either in the {\bnf rhs} or in the {\bnf lhs} or both.

\noindent
{\bnf eqns ::= eqn \{',' eqn \} '.'}\\
{\bnf eqn ::= lhs '=' rhs}\\
{\bnf lhs ::= exp}\\
{\bnf rhs ::= exp}\\
{\bnf exp ::= term | term '+' exp}\\
{\bnf term ::= number | coeff varid}\\
{\bnf coeff ::= empty | ['+'] number}\\
{\bnf varid ::= letter}\\

(Requirements Analysis: It is not clear if long equations are
presented on multiple lines in the input.  We will de facto allow it
as our grammar/parser of input takes care of it.
If our client insists that that should not be permitted, our
task becomes more complex.  The required grammar is no longer
context-free. )

{\small Other than using some kind of grammar, I know of no way to
describe this precondition.  Syntax diagrams etc. are, of course,
other ways of specifying context-free grammars.  Why should we let
{\bnf coeff} begin with an optional plus?  Is the grammar
``unambiguous'' ?  What does unambiguous mean in this context?  The
same as in an ``unambiguous specification'' ?  If duplicate-var-id
terms are not to be permitted in an equation, why does the required
grammar become more complex?  For that matter, are Pascal, C, Ada,
... context-free?  }

\subsection{Output is Equivalent to the Input}

{\small Item 3 was easier to do than item 2.  So we do that
next.}

The output is also a sentence derivable from {\bnf eqns}.  The $i$-th
equation of the input is {\sl semantically equivalent} to the $i$-th
equation of output.  Two equations $e_1 = e_2$ and $e_3 = e_4$ are
equivalent if $e_1$ is equivalent to $e_3$, and $e_2$ is equivalent to
$e_4$.  An expression $e$ is equivalent to $f$ if both $e$ and $f$
have the same terms, same number times in each.  Note that
permutations of terms is allowed.  Simplification of terms is not even
attempted.


\subsection{Output is Well-Formatted}

(Requirements Analysis: Since we impressed our customer by remarking
that we will imagine the paper width to be infinity, we choose to
output each equation as one line, regardless of its length.)

Let $n$ be the number of lines in the output.

The output consists of equations printed one per line.  The
order of the equations is the same as that of input.

To describe the formatting of the output, imagine dividing the 2-d
output into columns of $n$ rows as follows.  Draw vertical
straight-lines, from the topmost line to the bottom-most spanning the
$n$ lines of output, immediately surrounding each variable id letter,
each \verb|" + "| and the \verb|" = "|.  Similarly draw a line to the
immediate right of each number, and one straight line at the leftmost
edge of the output.

The first (i.e., the leftmost) column must be a number-column.  Each
row in a column of numbers must be either a string of blanks, or a
right-justified number.

Each row of the equals-column must contain \verb|" = "|.  Each row of
each plus-column must contain either a \verb|" + "| or three blanks.
All the rows of a variable id column must be 
either one blank or contain some letter, and all these letters
must be the same.

{\small Should we also say ``no column is all blanks''.  I think not;
what say you?  What about ``The first (i.e., the leftmost) column is a
number-column'' ?  }


\section{A Precise Solution}

\parindent=0pt

{\small The following is a precise and rigorous spec.  Is it
error-free?  Is it complete?  Obviously, some of the sentences from
the preceding section need to be reproduced below if this section were
the only thing you wrote.  }

\subsection{Syntax of Equations}

Imagine ({\small better yet: You write it!}) a predicate {\sl parses-ok$(G,
N, s)$} that yields true iff the string $s$ is a sentence derivable from
the non-terminal $N$ in the grammar $G$.


Input must be a well-formed sentence of $G$:
parses-ok$(G,$ {\bnf eqns}, {\tt input}$)$.


Let us call the required program as {\tt TE}.  Its signature is:

{\tt function TE(fi: seq of eqn) returns fo: seq of line}

\noindent
where {\tt eqn} is a set of sentences generated by the {\bnf eqn} of
the grammar above.

{\small The auxiliary function that obtains the \ffi\ and \ffo\ sequences
from their derivation trees is left as an exercise to you.  Note that
while the commas and the period in the input are helpful in parsing,
the \ffi\ sequence does not contain them; but the \ffo\ does.  }

\subsection{Output is Equivalent to the Input}

Two equations $e: e_1 = e_2$ and $e': e_3 = e_4$ are {\sl semantically
equivalent},

semantically-eq$(e, e') \zdef$\\
bag-of-terms$(e_1) = $ bag-of-terms$(e_3)$, \\
bag-of-terms$(e_2) = $ bag-of-terms$(e_4)$.

semantically-equal-eqns$($input, output$) \zdef$\\
\forall $i: 1..n$ $($
 parses-ok$(G,$ {\bnf eqn}, \ffo$[i])$ and
 semantically-eq$($\ffi$[i],$ \ffo$[i])$ $)$.

\noindent

{\small Note that in {parses-ok$(G,$ {\bnf eqn}, \ffo$[i])$}, it is \ffo\ and
not \ffi.  Why?  What is the signature of semantically-eq?  Is it
ok to write  semantically-eq$($ \ffi$[i]$, \ffo$[i])$ ?
}

\subsection{Output is Well-Formatted}

\subsubsection{There Exists a Matrix}

$\bullet$
We now imagine a matrix M of size $n \times c$, with the
following properties.

well-formatted$(M, c) \zdef$\\
\forall $y: 1..c$ $($ is-seq-chars$(y)$ and equal-width$(y)$ $)$ and \\
\forall $y: 1..c$
$($ is-a-plus$(y)$ or is-an-equals$(y)$ or is-a-varid$(y)$ or is-a-num$(y)$$)$

Recall that $n$ is the number of equations.  For now, we do not know
$c$; we simply postulate its existence.  We refer to $M[*, y]$ as the
y-th column.

We need the following aux functions.

$\bullet$
isblank$(s) \zdef $ \forall $i: 1..\#s (s[i] = $ {\rm blank} $)$.

$\bullet$
isnumber$(s) \zdef$ parses-ok$(G, number, s)$.

$\bullet$
rjustified$(s) \zdef$ \forsome  $k: 1..\#s$\\
$($ isblank$(s[1 .. k-1])$ and isnumber$(s[k .. k])$ and
isnumber$(s[k .. \#s])$ $)$

{\small Could we drop isnumber$(s[k .. k])$?
Should we change   isnumber$(s[k .. \#s])$ to isnumber$(s[k+1 .. \#s])$?}

$\bullet$
Each cell, $M[i, j]$, contains a seq of characters:\\
is-seq-chars$(y) \zdef $
\forall $i: 1..n$ $($ $M[i, y]$ in seq of char $)$.

$\bullet$
Each row in a given column $y$ is equal in
width to the others in that column:\\
equal-width$(y) \zdef $
\forall $i, j: 1..n$ $($ $\#M[i, y] = \#M[j, y]$ $)$ 

$\bullet$
\noindent
Column $y$ is a plus-column: \\
is-a-plus$(y) \zdef$\\
\forall $i: 1..n$
 $($ $M[i, y] = $ \verb|" + "| or isblank$(M[i, y])$ $)$ and \\
\forsome  $i: 1..n$ $($ $M[i, y] = $ \verb|" + "| $)$

$\bullet$
Column $y$ is an equals-column:\\
is-an-equals$(y) \zdef $
\forall $i: 1..n$ $($ $M[i, y]) = $ \verb|" = "| $)$

$\bullet$
Column $y$ is a varid column.  At least one row contains a letter.
Each row is either one blank or this letter.\\

is-a-varid$(y) \zdef $\\
\forsome  $z:$ letter\\
$($\quad \forsome  $x: 1..n$ $($ $M[x, y] = z$ $)$ and \\
.\quad \forall $i: 1..n$ $($ $M[i, y] = z$ or $M[i, y] = $ blank $)$ $)$

$\bullet$
Column $y$ is a number-column.  At least one row contains a number.
Each row is either all-blanks or contains a right-justified number:\\
is-a-num$(y) \zdef $\\
\forall $i: 1..n$ $($ isblank$(M[i, y])$ or rjustified$(M[i, y])$ $)$ and\\
\forsome  $i: 1..n$ $($ isnumber$(M[i, y])$ $)$

\subsubsection{Print the Rows of M}

The $i$-th line of output is a printing of the $i$-th row of M followed by
either a comma if $i < n$, or a period if $i = n$.

$\bullet$
print-all$(M) \zdef$\\
\forall $i: 1..n-1$ $($
\ffo$[i] = $ print$(i, M) \vbar $ {\tt ","} $)$ and
\ffo$[n] = $ print$(n, M) \vbar $ {\tt "."}


Print$(i, M)$ is simply a catenation of all the columns of the $i$-th
row and then trimming any blanks at the tail end.

{\small Define print$(i, M)$.
Shouldn't the previous section also say this?  Whatever happened to
the value $c$?  }

\subsubsection{Output is Looking Gooood!}

$\bullet$
looks-good$(s, n, c) \zdef$\\
\forsome $m:$ array $[1..n, 1..c]$ of seq of char\\
$($ well-formatted$(m, c)$ and $ s =$ print-all$(m)$ $)$

\subsection{Putting it all Together}

output $\zdef$ {\tt tabulateeqns}$($ input $)$, where\\
tabulateeqns$($input$) \zdef $ {\tt TE}$(q)$, where $q \zdef$
convert-to-eqn-seq$($input$)$.

{\small convert-to-eqn-seq$($input$)$ is a companion to
parses-ok.}

\noindent
This output is such that\\
semantically-equal-eqns$($input, output$)$ and\\
\forsome $c:$ nat $($ looks-good$($output$, \#q, c)$ $)$.

\section{Discussion}

The following are some imagined questions from students.

{\sl If we have some 40 minutes to spend on this problem, do you
expect us to finish it this well?}

{\small
Well, if you want an A, yes! More seriously, an answer along the lines
of Section 2 is certainly expected.  Section 3 is home work that you
should be able to do a good draft in a few, say 4, hours.  Such a
draft probably contains errors.  It is best to let others read it.  If
that is not possible, read it yourself, but after a day or two.}

{\sl Is it worth spending this much effort in specifying instead of
producing several hundreds of lines of code?}

{\small I think so.  There are no statistics that can validate or
invalidate this belief.  Assuming that the program being developed is
not a throw-away program, but has a long lifetime, the careful
translation of requirements to specs and then paraphrasing them back
to the user/client is an effective technique to prevent expensive
design and coding errors.}


%{\vfill\tiny Tue Nov 22 17:05:00 1994}
\end{document}
