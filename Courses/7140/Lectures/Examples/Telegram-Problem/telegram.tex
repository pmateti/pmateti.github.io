
Originally described by PeterNaur.  Write a program that takes a number
w, then accepts lines of text and outputs lines of text, where the
output lines have as many words as possible but are never longer than w
characters. Words may not be split, but you may assume that no single
word is too long for a line.

--

McKeeman,W. M., Respecifying the Telegram Problem, Report Date : 02 FEB
1977 Pagination or Media Count : 16 Abstract : The telegram problem was
initially proposed as a programming exercise suitable for beginning
students. A number of solutions have been published, all containing
errors. In this paper the trouble is traced to the original
specification and offer a new version. The most significant difference
is the use of a grammatical notation for some parts of the
specification. The new specification is then followed to a new
solution. The concusion is that the new specification is of higher
quality, but not different in kind from the original.

--

The program's input is a stream of characters whose end is signaled
with a special end-of-text character, ET.  There is exactly one ET
character in each input stream.  Characters are classified as

break characters -- BL (blank) and NL (new line);\\
nonbreak characters -- all others except ET;\\
the end-of-text indicator -- ET.\\

A {\sl word} is a nonempty sequence of nonbreak characters.  A {\sl break}
is a sequence of one or more break characters.  Thus, the input can be
viewed as a sequence of words separated by breaks, with possibly
leading and trailing breaks, and ending with ET.

The program's output should be the same sequence of words as in the
input, with the exception that an oversized word (i.e., a word
containing more than MAXPOS characters, where {\tt MAXPOS} is a
positive integer) should cause an error exit from the program (i.e., a
variable Alarm, should have the the value {\tt TRUE}).  Upto the point
of error, the program's output should have the following properties:

\begin{enumerate}
\item
A newline should start only between words and at the beginning of the
output text, if any.

\item
A break in the input is reduced to a single break character in the
output.

\item
As many words as possible should be placed on each line (i.e., between
successive
NL characters).

\item
No line may contain more than MAXPOS characters (words and BLs)
\end{enumerate}
